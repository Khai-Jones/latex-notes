\newpage

\lecturetitle{8}{Macroprudential Policy and Financial Stability}


\section{Introduction to Macroprudential Policy}

Macroprudential policy is a regulatory approach focused on the financial system as a whole, aiming to mitigate risks that could harm the overall economy. It emerged as a key policy area, particularly after the Global Financial Crisis highlighted the limitations of focusing solely on individual institutions.

\subsection{Definition and Primary Objective}
The primary objective of macroprudential policy is to \textbf{limit systemic risk}, which is the risk of disruptions to financial services that are caused by an impairment of all or parts of the financial system and have the potential to cause serious negative consequences for the real economy.

\subsection{Necessity and Goals}
Macroprudential policy is necessary to:
\begin{itemize}
    \item \textbf{Limit the risks and costs of systemic crises:} Prevent widespread financial instability and minimize associated economic disruptions.
    \item \textbf{Increase resilience to aggregate shocks:} Build financial buffers (like capital and liquidity) during good times that can be released during stress periods to absorb losses and maintain lending.
    \item \textbf{Contain the build-up of systemic vulnerabilities:}
        \begin{itemize}
            \item Address the 'time dimension' of risk: Reduce procyclical feedback between asset prices and credit, control unsustainable increases in leverage and debt (dampening financial imbalances and booms).
            \item Address the 'cross-sectional dimension' of risk: Manage structural vulnerabilities arising from interlinkages, common exposures, risk concentrations, and the critical role of specific institutions (Systemically Important Financial Institutions - SIFIs).
        \end{itemize}
\end{itemize}
Compared to monetary policy (often too blunt) and fiscal policy (often less flexible), macroprudential tools can be more tailored to specific risks, sectors, or loan portfolios, making them a complementary policy lever. The lack of a macroprudential perspective was identified as a key factor contributing to the Global Financial Crisis.

\section{Macroprudential vs. Microprudential Policy}

Macroprudential policy complements, but differs significantly from, traditional microprudential policy.

\begin{itemize}
    \item \textbf{Focus:}
        \begin{itemize}
            \item \textbf{Microprudential:} Safety and soundness of individual financial institutions. Aims to protect depositors/policyholders by reducing the probability of individual firm failure.
            \item \textbf{Macroprudential:} Stability of the financial system as a whole. Aims to protect the economy from system-wide financial distress.
        \end{itemize}
    \item \textbf{Risk Consideration:}
        \begin{itemize}
            \item \textbf{Microprudential:} Treats risks as largely exogenous (outside the control of the firm). Focuses on idiosyncratic risk of individual firms.
            \item \textbf{Macroprudential:} Treats risks as potentially endogenous (influenced by the collective behavior of firms). Focuses on systemic risk, including contagion, spillovers, and common exposures. Considers how actions rational for individual firms might collectively create system-wide problems (the \textbf{fallacy of composition}).
        \end{itemize}
    \item \textbf{Framework:}
        \begin{itemize}
            \item \textbf{Microprudential:} Often uses a partial equilibrium framework (analyzing a single firm in isolation).
            \item \textbf{Macroprudential:} Uses a system-wide, general equilibrium perspective, considering interactions and feedback loops.
        \end{itemize}
\end{itemize}
Macroprudential policy acts as an overlay to microprudential regulation, addressing risks that microprudential policy alone cannot sufficiently mitigate.

\section{Intermediate Objectives of Macroprudential Policy (ECB Example)}

The European Central Bank (ECB) outlines several intermediate objectives to guide its macroprudential policy actions:
\begin{enumerate}
    \item \textbf{Mitigate and prevent excessive credit growth and leverage:} Addresses the time dimension of systemic risk.
    \item \textbf{Mitigate and prevent excessive maturity mismatch and market illiquidity:} Focuses on structural funding vulnerabilities.
    \item \textbf{Limit direct and indirect exposure concentrations:} Addresses structural vulnerabilities from interconnectedness.
    \item \textbf{Limit the systemic impact of misaligned incentives} (with a view to reducing moral hazard): Controls structural vulnerabilities related to institutional behavior (e.g., remuneration, risk-taking).
\end{enumerate}

\section{Macroprudential Indicators}

Identifying and measuring systemic risk requires monitoring a range of indicators:

\begin{itemize}
    \item \textbf{Linked to Intermediate Objectives:}
        \begin{itemize}
            \item \textit{Excessive Credit/Leverage:} Credit-to-GDP gap (deviation from trend), household/corporate credit growth, asset prices (especially housing), leverage ratios.
            \item \textit{Maturity Mismatch/Illiquidity:} Structural funding ratios (e.g., Net Stable Funding Ratio - NSFR), liquidity stress indicators (e.g., Liquidity Coverage Ratio - LCR).
            \item \textit{Exposure Concentration:} Metrics related to size, complexity, substitutability, and interconnectedness of SIFIs; large exposure data.
            \item \textit{Misaligned Incentives:} Indicators related to compensation structures, risk appetite (often qualitative or under development).
        \end{itemize}
    \item \textbf{Timing Focus:}
        \begin{itemize}
            \item \textbf{Slow-moving leading indicators:} Signal risks building up over the medium term (e.g., credit-to-GDP gap, asset price bubbles). Often provide signals 2-4 years in advance. Supervisory information is also key here.
            \item \textbf{High-frequency market-based indicators:} Can help predict the imminent unwinding of systemic risk or measure current stress/interconnectedness (e.g., LIBOR-OIS spread, yield curve shape, CDS spreads, equity correlations).
        \end{itemize}
\end{itemize}
No single indicator is sufficient; policymakers use a suite of indicators, often presented in 'heat maps', and apply judgment to assess overall systemic risk. Distinguishing between benign ("good") shocks and potentially harmful ("bad") shocks is crucial.

\section{Macroprudential Instruments}

Authorities use a variety of tools, often existing prudential instruments calibrated with a systemic focus, to achieve macroprudential objectives. These instruments require appropriate governance arrangements.

\subsection{Capital-Related Measures}
These tools aim to increase the banking system's loss-absorbing capacity.
\begin{itemize}
    \item \textbf{Countercyclical Capital Buffer (CCyB):}
        \begin{itemize}
            \item \textit{Objective:} Build capital defences during periods of excessive credit growth (often linked to economic booms) to enhance resilience and control the financial cycle. Prevent systemic risk emergence.
            \item \textit{Mechanism:} Requires banks to accumulate additional Common Equity Tier 1 (CET1) capital (typically 0-2.5\% of Risk-Weighted Assets under Basel III, though national authorities can go higher) when systemic risk is judged to be building. This buffer can then be released during stress periods, allowing banks to absorb losses and maintain lending without breaching minimum requirements.
            \item \textit{Activation/Calibration:} Often guided by the credit-to-GDP gap, but also considers other indicators (asset prices, leverage) and requires significant judgment. National authorities decide on activation and release based on sound, transparent principles.
            \item \textit{Benefits:} Reduces crisis probability/severity, dampens financial/real cycles, mitigates credit crunches.
            \item \textit{Costs:} May raise lending costs (regulatory spread), potentially lower output/consumption growth in the medium term, risk of activity shifting to unregulated sectors (leakage).
            \item \textit{Interaction:} Acts as a macroprudential overlay within Basel III, complementing the permanent Capital Conservation Buffer.
        \end{itemize}
    \item \textbf{Sectoral Capital Requirements/Risk Weights:} Higher capital requirements or risk weights applied to exposures deemed systemically risky (e.g., specific types of mortgages).
    \item \textbf{Capital Surcharges for Systemically Important Financial Institutions (SIFIs):} Higher capital requirements for institutions whose failure would pose a greater risk to the system.
    \item \textbf{Contingent Capital (CoCos):} Debt instruments that convert to equity or are written down when a pre-specified trigger event occurs (e.g., bank's capital falls below a certain level).
\end{itemize}

\subsection{Credit-Related Measures}
These tools target borrowers or lending standards directly.
\begin{itemize}
    \item \textbf{Caps on Loan-to-Value (LTV) Ratios:} Limits the loan amount relative to the value of the collateral (e.g., house price).
    \item \textbf{Caps on Debt-to-Income (DTI) or Debt-Service-to-Income (DSTI) Ratios:} Limits the borrower's total debt or debt payments relative to their income.
    \item \textbf{Limits on Credit Growth:} Direct quantitative restrictions on overall or sectoral credit expansion.
    \item \textbf{Restrictions on Foreign Currency (FX) Lending:} Limits or stricter conditions on loans denominated in a foreign currency, particularly to unhedged borrowers.
    \item \textbf{Dynamic Loan Loss Provisioning:} Requires banks to build up provisions for loan losses during good times based on expected future losses (forward-looking), rather than just incurred losses.
    \item \textbf{Minimum Haircuts/Margins for Secured Lending:} Requirements for collateral value to exceed loan value by a certain margin in secured funding markets (e.g., repo).
\end{itemize}

\subsection{Liquidity-Related Measures}
These tools aim to mitigate funding risks.
\begin{itemize}
    \item \textbf{Liquidity Coverage Ratio (LCR):} Requires banks to hold sufficient high-quality liquid assets (HQLA) to withstand a 30-day stressed funding scenario (Basel III).
    \item \textbf{Net Stable Funding Ratio (NSFR):} Requires banks to maintain a stable funding profile over a one-year horizon based on the liquidity characteristics of their assets and liabilities (Basel III).
    \item \textbf{Limits on Net Open Currency Positions/Currency Mismatch:} Restricts exposure to foreign exchange risk.
    \item \textbf{Restrictions on Funding Sources:} Limits on reliance on volatile wholesale funding.
    \item \textbf{Reserve Requirements:} Requiring banks to hold a certain percentage of deposits as reserves (can sometimes be used with a macroprudential objective, e.g., differentiated for FX liabilities).
\end{itemize}

\subsection{Interconnectedness Measures}
These tools target risks arising from links between institutions.
\begin{itemize}
    \item \textbf{Limits on Interbank Exposures / Large Exposures:} Restricting the amount one financial institution can lend to or hold claims on another.
    \item \textbf{Regulations on Shadow Banking:} Rules addressing risks in non-bank financial intermediation that performs bank-like functions.
    \item \textbf{Trading Infrastructure Recommendations/Guidelines:} Standards for central counterparties (CCPs), trade repositories, etc., to enhance market resilience.
    \item \textbf{Securities Disclosure Recommendations:} Improving transparency in complex structured products.
\end{itemize}

\subsection{Other Tools}
\begin{itemize}
    \item \textbf{Taxation of Capital Flows or Non-Deposit Liabilities:} Levies designed to discourage potentially destabilizing inflows or funding structures.
    \item \textbf{Restrictions on Profit Distribution:} Limiting dividend payouts or share buybacks to conserve capital.
\end{itemize}

\section{How Macroprudential Instruments Work (Channels of Effectiveness)}

Macroprudential tools influence the financial system through several channels:
\begin{itemize}
    \item \textbf{Reducing Excessive Credit Growth and Leverage:} Directly via LTV/DTI caps or credit limits; indirectly via capital requirements (CCyB) making lending more expensive.
    \item \textbf{Building Financial Resilience:} Increasing loss-absorption capacity through higher capital (CCyB, SIFI surcharges) and liquidity buffers (LCR, NSFR), enabling institutions to withstand shocks and continue lending.
    \item \textbf{Limiting Procyclicality:} Building buffers in good times and releasing them in bad times (CCyB, dynamic provisioning) dampens the amplification of financial and business cycles.
    \item \textbf{Managing Specific Vulnerabilities:} Targeted instruments address risks like FX lending or maturity mismatches.
    \item \textbf{Influencing Expectations:} Clear communication of objectives, risk assessments, and policy strategies guides market behavior and reduces uncertainty.
    \item \textbf{Addressing Interconnectedness and Risk-Taking Incentives:} Limits on exposures reduce contagion risk. Higher capital requirements and LTV/DTI caps encourage more prudent lending standards and reduce moral hazard by internalizing some social costs of excessive risk-taking.
\end{itemize}

\section{Interaction with Monetary Policy}

Monetary and macroprudential policies interact significantly and require coordination.
\begin{itemize}
    \item \textbf{Monetary Policy Side Effects:} Accommodative monetary policy (low interest rates) can contribute to financial imbalances by encouraging borrowing (lower debt service costs, higher asset prices as collateral) and potentially excessive risk-taking (search for yield).
    \item \textbf{Macroprudential Policy Dampening Effects:} Macroprudential tools (LTV/DTI caps, CCyB) can mitigate these side effects, allowing monetary policy to focus on its primary mandate (often price stability) without unduly jeopardizing financial stability.
    \item \textbf{Macroprudential Tightening Effects:} Tightening macroprudential policy can dampen credit growth and output. Monetary policy might need to adjust to offset these effects, depending on the overall economic situation.
    \item \textbf{Enhancing Monetary Policy Effectiveness:} By building capital and liquidity buffers, macroprudential policy strengthens financial system resilience. During stress, releasing these buffers helps maintain credit flow and keeps the monetary policy transmission mechanism open, reducing the need for aggressive monetary easing and lowering the risk of hitting the zero lower bound on interest rates or relying heavily on non-traditional policies.
    \item \textbf{Non-traditional Monetary Policy:} Accommodative stances via non-traditional tools (QE, negative rates) can also incentivize risk-taking, increasing the importance of macroprudential backstops. Effective macroprudential policy ex-ante can reduce the need for extensive non-traditional easing ex-post during crises.
\end{itemize}

\section{Policy Coordination and Institutional Arrangements}

Effective macroprudential policy requires careful coordination and appropriate institutional structures.
\begin{itemize}
    \item \textbf{Coordination Challenges:} Monetary and macroprudential policies can have overlapping effects and potential conflicts (e.g., low rates for price stability might fuel financial imbalances). Coordination with microprudential and fiscal policy is also crucial.
    \item \textbf{Institutional Arrangements:} Essential elements include a clear mandate, operational independence, adequate powers (information gathering, instrument deployment), strong accountability, and effective communication strategies. The specific setup (e.g., within central bank, separate committee) varies by country. Mechanisms for conflict resolution may be needed.
    \item \textbf{Political Economy:} Macroprudential tools can be targeted and face political opposition from affected sectors, potentially impacting policy independence and requiring strong institutional backing. Establishing frameworks is often a lengthy process.
\end{itemize}

\section{Regulatory Structures and Implementation}

The structure of financial regulation impacts how macroprudential policy is implemented.
\begin{itemize}
    \item \textbf{Regulatory Models:}
        \begin{itemize}
            \item \textbf{Twin Peaks Model:} Separates prudential supervision (safety and soundness of firms) and conduct-of-business supervision (market integrity, consumer protection) into distinct authorities (e.g., Netherlands DNB/AFM). Macroprudential oversight is often integrated within or closely linked to the prudential authority.
            \item \textbf{Single Peak (Unified) Model:} A single authority handles both prudential and conduct supervision (e.g., UK FSA prior to reform).
        \end{itemize}
    \item \textbf{Impact on Intermediaries:} Prudential rules (capital, liquidity, risk management) make intermediaries resilient. Conduct rules shape interactions with customers and markets. Macroprudential tools (CCyB, LTV/DTI) overlay these, influencing lending capacity, costs, and risk models based on systemic considerations.
    \item \textbf{Implementation Approach:} Translating risk assessments into policy actions involves designing and calibrating tools. A "guided discretion" approach, combining rules/indicators with judgment, is common. Addressing potential leakage (risk migrating to less regulated sectors) is a key challenge.
\end{itemize}

\section{Implementation Challenges and Considerations}

\begin{itemize}
    \item \textbf{Monitoring and Detection:} Developing robust methods and indicators to monitor risk buildup and detect imminent threats.
    \item \textbf{Tool Design and Calibration:} Creating effective policy tools and calibrating them appropriately.
    \item \textbf{Trade-offs and Costs:} Balancing financial stability goals with potential negative impacts on economic activity, regulatory burden, and market distortions.
    \item \textbf{Cross-Border Issues:} In integrated economies, policies face leakages (activity moving abroad), spillovers (effects on other countries), and positive externalities (benefits of stability spilling over). International cooperation and consistency are vital.
    \item \textbf{Scope:} Monitoring and potentially extending regulation to cover systemic risks arising outside the traditional banking system (e.g., shadow banking, capital flows, securitization markets).
\end{itemize}

\section{Conclusion}

Macroprudential policy has become an essential pillar of financial regulation, aiming to safeguard the stability of the entire financial system. By addressing the build-up and transmission of systemic risk through various indicators and instruments, it complements microprudential regulation and interacts significantly with monetary policy. While challenges in coordination, implementation, calibration, and managing cross-border effects remain, macroprudential policy is crucial for strengthening the financial system's defences against shocks and ensuring it can continue to support sustainable economic growth.
