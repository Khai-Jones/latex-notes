\newpage

\lecturetitle{2}{Theories of Financial Intermediation}


\subsection*{What is a Financial Intermediary?}

\begin{itemize}
    \item ``An Economic agent which specializes in the activities of buying and selling (usually at the same time) financial claims.''
    \item This definition is also linked to the idea of an intermediary seen in industrial organization as:
    \item ``An agent who buys certain goods or services from producers and sells them to final consumers.''
    \item Banks can be described as retailers of financial securities – buying securities issued by borrowers (extending loans) and selling to lenders (collecting deposits).
\end{itemize}

Banks tend to focus on or execute transactions primarily related to financial contracts (loans and deposits) as opposed to financial securities (stocks and bonds). Thus, these contracts must be held on balance sheets until expiration. Key theoretical areas underpinning financial intermediation include: Information Asymmetry, Transaction Costs and Efficiency, Liquidity and Risk Management, Regulatory Constraints, and Economic Development.

\subsection{Theories Based on Information Asymmetry}
\begin{itemize}
    \item \textbf{Delegated Monitoring Theory (Diamond, 1984):} Banks act as monitors to reduce information asymmetry and prevent borrower opportunism.
    \item \textbf{Adverse Selection Theory (Akerlof, 1970; Stiglitz \& Weiss, 1981):} Financial intermediaries screen borrowers to mitigate hidden risk before lending.
    \item \textbf{Moral Hazard Theory (Holmström \& Tirole, 1997):} Banks monitor borrowers post-lending to reduce risk-taking behaviour.
    \item \textbf{Credit Rationing Theory (Stiglitz \& Weiss, 1981):} Lenders ration credit rather than increasing interest rates due to asymmetric information.
\end{itemize}

\subsection{Theories Based on Transaction Costs and Efficiency}
\begin{itemize}
    \item \textbf{Transaction Cost Theory (Benston \& Smith, 1976):} Financial intermediaries lower transaction costs by centralizing lending, borrowing, and payments.
    \item \textbf{Economies of Scale in Banking (Klein, 1971):} Intermediaries reduce per-unit costs of financial services by spreading fixed costs over a larger volume.
    \item \textbf{Bank-Based vs. Market-Based Intermediation (Levine, 2002):} Efficiency varies between bank-centric and market-centric financial systems.
\end{itemize}

\subsection{Theories Based on Liquidity and Risk Management}
\begin{itemize}
    \item \textbf{Liquidity Transformation Theory (Diamond \& Dybvig, 1983):} Banks convert short-term deposits into long-term loans, managing liquidity risk.
    \item \textbf{Risk Transformation Theory:} Banks pool and redistribute risk, offering diversification benefits to depositors.
    \item \textbf{Financial Accelerator Theory (Bernanke, Gertler \& Gilchrist, 1999):} Credit cycles amplify economic fluctuations due to shifts in financial conditions.
\end{itemize}

\subsection{Theories Based on Regulatory Constraints and Policy Implications}
\begin{itemize}
    \item \textbf{Financial Repression Theory (McKinnon \& Shaw, 1973):} Government-imposed restrictions (e.g., interest rate caps) distort financial intermediation.
    \item \textbf{Too Big to Fail and Moral Hazard (Kane, 1989):} Implicit government guarantees create risk-taking incentives for large financial institutions.
    \item \textbf{Regulatory Arbitrage Theory:} Banks exploit differences in regulations to maximize profits, often shifting risks outside the regulated system.
\end{itemize}

\subsection{Theories Linking Financial Intermediation to Economic Development}
\begin{itemize}
    \item \textbf{Theory of Financial Intermediation and Economic Growth (Schumpeter, 1911; Levine, 1997):} Financial intermediaries promote growth by allocating capital efficiently.
    \item \textbf{Financial Liberalization and Development (McKinnon \& Shaw, 1973):} Market-based intermediation fosters economic efficiency and development.
\end{itemize}

\section{Delegated Monitoring}

Understanding this theory elucidates the role of financial intermediaries, particularly banks, in mitigating information asymmetries between borrowers and lenders.

\begin{itemize}
    \item \textbf{Information Asymmetry:} Borrowers possess more information about their investment projects than lenders, leading to potential adverse selection and moral hazard.
    \item \textbf{Monitoring Costs:} Direct monitoring by individual lenders is costly and inefficient due to duplication of efforts.
    \item \textbf{Delegation to Intermediaries:} Financial intermediaries, such as banks, act as delegated monitors, pooling resources to efficiently oversee borrowers on behalf of individual lenders.
    \item \textbf{Diversification:} By diversifying their loan portfolios, intermediaries reduce the risk of borrower default, enhancing their monitoring efficiency.
\end{itemize}

Monitoring here is referred to broadly in the following ways:
\begin{itemize}
    \item \textbf{Screening:} inserts an a priori element in an adverse selection context (Broecker 1990);
    \item \textbf{Preventing:} reduces or prevents opportunistic behaviour during the realization of the project (Holmstrom and Tirole 1997);
    \item \textbf{Punishing or auditing:} occurs when a borrower fails to meet contractual obligations - and becomes costly in the context of state verification.
\end{itemize}

The theory overall suggests that financial intermediaries such as banks have a comparative advantage in monitoring activities. If they do have an advantage, then several tenets are necessary:
\begin{itemize}
    \item \textbf{Economies of Scale in monitoring:} This suggests that an FI has the ability to finance many projects.
    \item \textbf{Limited capacity of investors:} When compared with the size of the projects, this element implies that each project needs the resources of several investors.
    \item \textbf{Low costs of Delegation:} The cost of monitoring or controlling the FI is less than the surplus gained from exploiting scale economies in monitoring or controlling investment projects.
\end{itemize}

\subsection*{Divergences from the Theory}
\begin{itemize}
    \item \textbf{Direct Lending:} In some markets, lenders may choose to lend directly to borrowers without involving intermediaries, especially when:
        \begin{itemize}
            \item The cost of monitoring is low.
            \item Lenders have sufficient expertise to assess and manage risks.
        \end{itemize}
    \item \textbf{Peer-to-Peer Lending Platforms:} These platforms connect lenders and borrowers directly, utilizing technology to reduce information asymmetries and monitoring costs. However, they may lack the diversification benefits and risk management expertise of traditional financial intermediaries.
\end{itemize}

\section{Transaction Costs Theory}

The transaction cost approach to the theory of financial intermediation focuses on how financial intermediaries reduce the various costs associated with financial transactions. This approach views intermediaries as mechanisms that lower the frictions in the financial system, making it more efficient for both savers and borrowers.

\begin{itemize}
    \item \textbf{Cost Reduction:} The transaction cost approach posits that financial intermediaries exist to minimize the expenses associated with financial activities. These costs can include those incurred when making transactions, searching for suitable counterparties, and monitoring investments.
    \item \textbf{Economies of Scale and Scope:} Financial intermediaries can achieve economies of scale by spreading the fixed costs of asset evaluation, information gathering, and transaction processing over a larger number of transactions. This reduces the cost per transaction, making it more efficient than if individuals tried to perform these tasks on their own. They can also achieve economies of scope by offering a range of services and products to customers.
    \item \textbf{Market Making:} Financial intermediaries, such as market makers and dealers, play a crucial role in reducing information costs by providing a marketplace where potential buyers and sellers can meet. By doing so, they lower the costs associated with searching for counterparties. They also take positions in assets, providing liquidity to the market.
    \item \textbf{Search Costs:} Intermediaries reduce search costs associated with finding a suitable party for a transaction. Instead of an individual searching for a lender or a borrower, an intermediary creates a market where these parties can easily transact.
    \item \textbf{Monitoring and Auditing Costs:} Financial intermediaries lower the costs of monitoring and auditing borrowers by acting as delegated monitors, specializing in overseeing the behaviour of borrowers. They develop expertise in evaluating and monitoring borrowers, making this process more efficient and less costly than if individual savers attempted to do it themselves.
    \item \textbf{Transactions Technology:} The transaction costs approach considers that the financial intermediaries act as coalitions of individual lenders or borrowers who exploit economies of scale or scope in the transaction technology. The notion of transaction costs includes not only exchange or monetary transaction costs but also search and monitoring costs.
\end{itemize}

\subsection*{Divergences from the Theory}
\begin{itemize}
    \item \textbf{Risk Management as a Primary Driver:} A major divergence is the argument that risk management, rather than solely transaction cost reduction, is the primary driver of financial intermediation. This perspective suggests that financial intermediaries are not just agents that reduce transaction costs but are active participants that absorb and transform risks. They bridge the gap between risk-averse savers and risk-seeking investors by offering products that cater to different risk preferences. This view posits that financial intermediaries create value by managing various forms of risk, such as maturity risk, counterparty risk, and market risk.
    \item \textbf{Regulation and its Endogenous Nature:} The transaction cost approach often considers regulation as an exogenous factor that creates market imperfections. However, some argue that regulation is endogenous, meaning it's influenced by and in turn influences the financial industry. Regulation may generate rents for regulated intermediaries, and financial institutions may try to circumvent or adapt to regulations.
    \item \textbf{The Role of Value Creation:} A criticism of the transaction cost approach is that it portrays financial intermediaries as merely responding to market imperfections. An alternative view emphasizes the value creation aspect of financial intermediation. This perspective posits that intermediaries actively create new financial instruments and markets by meeting the specific needs of savers and investors. They transform savings into investments and develop customized financial solutions, which cannot always be created by savers and investors on their own.
\end{itemize}

\section{Liquidity Transformation Theory (Diamond-Dybvig Model)}

The Diamond-Dybvig model of financial intermediation focuses on how banks transform illiquid assets into liquid liabilities, addressing the demand for liquidity in an economy with uncertain consumption needs. This model provides an explanation for why banks, despite being vulnerable to runs, are able to attract deposits and perform a valuable economic function.

\subsection{The Demand for and Provision of Liquidity}
\begin{itemize}
    \item \textbf{Demand for Liquidity and Uncertainty:} The model starts with the premise that individuals encounter risks that are privately observed, leading to a need for liquidity. Agents are uncertain about their consumption requirements and recognize them in the first period. Type 1 agents focus solely on their consumption during this initial period, while type 2 agents prioritize consumption in the second period. This uncertainty necessitates a mechanism that enables them to access consumption at critical moments, which is liquidity.
    \item \textbf{Role of Banks as Liquidity Providers:} Banks address this need by converting illiquid assets into liquid liabilities. They engage in long-term capital investments that are somewhat irreversible and provide demand deposits. These deposits are classified as liquid liabilities since they allow depositors to withdraw their funds whenever needed, while the bank’s assets (long-term investments) are not easily liquidated into cash.
    \item \textbf{Role of Insurance:} Banks play a crucial role by providing insurance that ensures investors receive a return when cashing in before maturity. This feature enables agents to consume at critical moments, which is essential for effective risk-sharing within the economy. Banks achieve this by converting illiquid assets into liquid liabilities.
    \item \textbf{Importance of Transformation:} The process of turning illiquid assets into liquid liabilities is significant as it promotes a more consistent return pattern over time compared to what illiquid assets can provide. However, this transformation carries its own set of risks: by providing liquid liabilities, banks expose themselves to the possibility of runs.
\end{itemize}

\subsection{Vulnerability to Bank Runs}
\begin{itemize}
    \item \textbf{Multiple Equilibria:} The transformation of illiquid assets into liquid liabilities creates multiple equilibria. One is an efficient risk-sharing equilibrium, and another is a bank run.
    \item \textbf{Efficient Equilibrium:} If depositors are confident in the bank, they will withdraw only when they need to under optimal risk-sharing. A withdrawal signals that a depositor should withdraw under optimal risk sharing.
    \item \textbf{Bank Run Equilibrium:} If depositors panic, they will rush to withdraw their deposits before the bank runs out of assets, regardless of their true need.
    \item \textbf{Self-Fulfilling Prophecy:} A bank run can manifest as a self-fulfilling prophecy, where the mere anticipation of a run triggers its occurrence. This can transpire even when the bank is financially sound, as the process of liquidating assets leads to losses.
    \item \textbf{Forced Liquidation:} A bank run compels the institution to liquidate its assets, often at a loss, even if not all depositors choose to withdraw their funds. The bank is required to sell all its assets, resulting in losses during the liquidation process.
    \item \textbf{Real Economic Damage:} Bank runs inflict genuine economic harm rather than merely indicating existing issues. The necessity to liquidate assets at a loss during a run diminishes overall economic value.
\end{itemize}

\subsection{Key Mechanisms and Assumptions}
\begin{itemize}
    \item \textbf{Sequential Service Constraint:} Demand deposit contracts operate under a sequential service constraint. This indicates that the bank's payments to any depositor are determined solely by their position in the queue, rather than any future information. Depositors are served in a random sequence, mimicking continuous time where deposits and withdrawals occur at unpredictable intervals.
    \item \textbf{Asymmetric Information:} Asymmetric information underlies the necessity for liquidity. The agent's type is private knowledge, leading to a requirement for liquidity and creating conditions that may result in a bank run. With complete information, the demand for liquidity would diminish, thereby lowering the susceptibility of banks.
    \item \textbf{Irreversible Long-Term Investments:} The model posits that long-term capital investments are largely irreversible, a notion supported by any transaction costs incurred when selling assets prior to their maturity. Consequently, it becomes expensive for the bank to liquidate assets ahead of schedule.
\end{itemize}

\subsection{Implications and Extensions}
\begin{itemize}
    \item \textbf{Deposit Insurance:} Deposit insurance can eliminate the incentive for depositors to run on a bank, thus preventing a costly liquidation of assets.
    \item \textbf{Liquidity Crises Beyond Banking:} The model's insights extend beyond banking. Any firm that issues short-term bonds to finance illiquid assets can face a similar liquidity crisis.
    \item \textbf{Bankruptcy Laws:} The protection from creditors that bankruptcy laws provide serves a similar function as the suspension of convertibility because it allows a viable but illiquid firm to survive.
    \item \textbf{Liquidity Insurance:} Banks with deposit insurance can provide "liquidity insurance" to firms, preventing liquidity crises for firms with short-term debt. This limits the need for firms to use bankruptcy to stop such crises.
    \item \textbf{Focus on Intermediaries:} The model's focus on financial intermediaries is supported by the fact that banks directly hold a significant fraction of the short-term debt of corporations. This highlights that most of the aggregate liquidity risk is channeled through insured financial intermediaries.
\end{itemize}

\section{Financial Accelerator Theory}

The financial accelerator theory, articulated by Bernanke and colleagues, illustrates how disruptions in credit markets can enhance and transmit shocks throughout the macroeconomy. It highlights the significance of borrowers' net worth in influencing the external finance premium and its subsequent impact on investment and economic performance. The fundamental aspects of this theory can be categorized into several key themes.

\subsection{Credit Market Frictions and the External Finance Premium}
\begin{itemize}
    \item \textbf{Asymmetric Information:} The model is based on the idea that asymmetric information between borrowers and lenders is a key feature of credit markets. Lenders face costs of gathering information and monitoring borrowers, which leads to agency problems and an external finance premium.
    \item \textbf{External Finance Premium:} The external finance premium is the difference between the cost of funds raised externally and the opportunity cost of funds internal to the firm. This premium reflects the agency costs associated with lending.
    \item \textbf{Net Worth as a Mitigating Factor:} The net worth of potential borrowers, defined as their liquid assets plus the collateral value of illiquid assets less outstanding obligations, plays a critical role. Higher levels of net worth reduce the external finance premium because it mitigates the agency problems associated with external finance. When borrowers have more of their own wealth to contribute to project financing, the potential divergence of interests between the borrower and the suppliers of external funds is reduced, thus reducing agency costs and the premium on external finance.
    \item \textbf{Inverse Relationship:} There is an inverse relationship between a borrower's net worth and the external finance premium. Borrowers with lower net worth face a higher premium due to increased agency costs, while those with higher net worth face a lower premium.
\end{itemize}

\subsection{The Financial Accelerator Mechanism}
\begin{itemize}
    \item \textbf{Procyclical Net Worth:} Borrowers' net worth tends to be procyclical, increasing during economic expansions due to rising profits and asset prices, and decreasing during recessions.
    \item \textbf{Countercyclical External Finance Premium:} Because of the inverse relationship between net worth and the external finance premium, the premium is countercyclical: It decreases during expansions, encouraging borrowing and investment, and increases during recessions, discouraging borrowing and investment.
    \item \textbf{Amplification of Shocks:} The countercyclical characteristic of the external finance premium acts as a financial accelerator, magnifying the impact of economic shocks. When positive shocks occur, boosting asset prices and profits, net worth rises, which reduces the external finance premium. This encourages greater borrowing and investment, further enhancing economic growth. In contrast, negative shocks lead to a decline in asset prices, a drop in net worth, and an increase in the external finance premium, resulting in decreased borrowing and investment, which further contracts the economy.
    \item \textbf{Propagation of Shocks:} The mechanism of the financial accelerator also serves to propagate the effects of shocks over time. Since changes in net worth take time to return to their usual levels, the impact of these shocks on investment and output is distributed over several periods.
\end{itemize}

\subsection{Key Implications and Extensions}
\begin{itemize}
    \item \textbf{Monetary Policy Transmission:} The model demonstrates how credit market imperfections influence the transmission of monetary policy. Credit frictions can amplify the effects of monetary policy shocks. For example, a monetary easing leads to increased asset prices and net worth, which reduces the external finance premium, leading to more borrowing and investment.
    \item \textbf{Explanation of Business Cycles:} The financial accelerator helps to explain why relatively small shocks can have large real effects on the economy. The mechanism amplifies both real and nominal shocks and also helps explain the persistence of business cycles.
    \item \textbf{Role in Crises:} The financial accelerator mechanism provides a rationale for why financial crises can have devastating effects on the real economy. A collapse in asset prices reduces net worth, increases the external finance premium, and can lead to a sharp contraction in lending, investment, and output.
    \item \textbf{Importance of Net Worth:} The model underscores the importance of borrowers' net worth in macroeconomic dynamics. The endogenous evolution of net worth is central to the propagation of shocks.
\end{itemize}

\section{Financial Repression Theory}

The McKinnon-Shaw hypothesis suggests that financial repression, typically marked by low or negative real interest rates, obstructs financial intermediation and hampers economic growth. It can also arise when governmental policies maintain real interest rates at artificially low or negative levels. Such conditions may result from interest rate ceilings or various regulatory measures. This situation discourages saving and diminishes the availability of credit, as savers do not receive proper compensation for their funds. It highlights how government interventions that disrupt financial markets, like imposing interest rate ceilings, can hinder the effective distribution of capital. Below are the key aspects of the financial repression theory.

\subsection{The Role of Real Interest Rates}
\begin{itemize}
    \item \textbf{Real Interest Rates as a Key Indicator:} The McKinnon-Shaw hypothesis suggests that the level of financial intermediation is closely related to the prevailing level of the real interest rate. According to this view, the real interest rate indicates the extent of financial repression.
    \item \textbf{Positive Real Interest Rates:} The theory argues that a positive real interest rate stimulates financial savings and financial intermediation. This increased savings then increases the supply of credit to the private sector, which in turn stimulates investment and growth.
\end{itemize}

\subsection{Impact on Financial Intermediation}
\begin{itemize}
    \item \textbf{Savings and Credit Supply:} The main channel of transmission emphasized by the McKinnon-Shaw hypothesis is the effect of real interest rates on the volume of savings. Positive real interest rates encourage savings, increasing the supply of loanable funds. Conversely, low or negative real interest rates discourage savings, reducing the pool of funds available for lending.
    \item \textbf{Efficient Allocation of Funds:} The theory also posits that positive real interest rates make the allocation of investable funds more efficient, thus providing an additional positive effect on economic growth. When interest rates are allowed to reflect market conditions, funds tend to flow to the most productive uses.
    \item \textbf{Reduced Financial Intermediation:} Financial repression, by keeping interest rates low, hinders the efficient flow of funds from savers to borrowers. This can lead to credit rationing where funds may not be allocated to the most productive uses because there is not sufficient incentive to save and invest in the financial system.
\end{itemize}

\subsection{Relationship with Economic Growth}
\begin{itemize}
    \item \textbf{Stimulating Investment and Growth:} According to the McKinnon-Shaw hypothesis, a positive real interest rate stimulates financial savings and financial intermediation, which in turn increases the supply of credit to the private sector. This leads to increased investment and ultimately stimulates economic growth.
    \item \textbf{Negative Impact of Repression:} Financial repression, on the other hand, has a negative impact on economic growth by reducing the volume of savings and investment and by inefficient allocation of resources. When real interest rates are kept artificially low, there is less incentive to save, and credit may be misallocated, leading to lower overall growth.
    \item \textbf{Inflation's Role:} High inflation rates can correspond to low economic growth rates, further complicating the relationship between interest rates and financial intermediation. High inflation erodes the real return on savings, further reducing incentives to save and invest.
\end{itemize}
