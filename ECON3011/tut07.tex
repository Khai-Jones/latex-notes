\section{Problem Set 7}


1.Define systemic risk. How does it differ from idiosyncratic risk? Provide an example of each.

2. Explain two key channels through which systemic risk can propagate in a financial system. How might these channels interact during a crisis?

3. Describe how liquidity shortages at one institution can become a system-wide crisis. Refer to at least one historical example.

4. Discuss the role of leverage and procyclicality in amplifying systemic risk. What regulatory tools can be used to mitigate these effects?

5. What is the role of macroprudential regulation in reducing systemic risk? Mention at least two specific tools and explain their function.
