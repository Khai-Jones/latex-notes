\tableofcontents

\newpage

\lecturetitle{1}{Introduction to Financial Systems}


\section{The Global Financial System}

The global financial system is a complex network of financial institutions, markets, instruments, and regulations that operate internationally to facilitate the allocation of capital, risk-sharing, and financial services across borders. This system encompasses banks, capital markets, insurers, central banks, and international financial institutions, as well as regulatory frameworks that govern cross-border financial activities.

\section{The Financial System}

The financial system is complex, and comprises different types of private sector financial institutions. Financial Markets channel funds from households, firms, and governments to those who may have a shortage of funds. They do this through financial intermediaries and financial markets. Key institutions include:
\begin{itemize}
    \item Banks
    \item Credit Unions
    \item Insurance Companies
    \item Mutual Funds
    \item Pension Funds
    \item Finance Companies
\end{itemize}

\section{The Financial Systems of CARICOM}

The financial sector within CARICOM is predominantly bank-centric with other important sectors such as credit unions, insurance companies and pension funds. The banking sectors in the region are a mix of local and foreign-owned entities (usually subsidiaries of Canadian parent companies).

\section{The Function of Financial Markets}

\subsection{Direct Finance}
\begin{itemize}
    \item Borrowers borrow funds directly from lenders in financial markets by selling them securities/financial instruments (which are claims on the borrower’s future income or assets).
    \item \textbf{Why is this important?}
    \begin{itemize}
        \item Persons who save are not the same as the ones who may profit from investment opportunities.
        \item In the absence of financial markets, lenders and borrowers are unlikely to meet/connect.
    \end{itemize}
\end{itemize}
\textit{The principal lender-savers are households, while governments, businesses, and foreigners may also have additional funds to lend out. On the other hand, the most important borrower-spenders are businesses and the government.}

\subsection{Indirect Finance}
\begin{itemize}
    \item The Financial Intermediary (FI) borrows funds from lenders-savers and then makes loans to borrowers/spenders.
    \item This process is called \textbf{financial intermediation} and is the primary route for moving funds between lenders and borrowers.
    \item In considering indirect finance, we must also consider transaction costs, risk sharing, and information costs in financial markets.
\end{itemize}
\textit{Funds may also move from lenders to borrowers through indirect finance – because it relies on a financial intermediary (FI) to facilitate the transaction.}

\section{The Structure of Financial Markets}

\subsection{Debt and Equity Markets}
A firm or individual may obtain funds in a financial market in two ways. The most common method is to issue a \textbf{debt instrument}, such as a bond or a mortgage. The maturity of a debt instrument is number of years (term) until that instrument’s expiration date. Alternatively, they can issue \textbf{equity}, such as common stock, which are claims to share in the net income and assets of a business.

\subsection{Primary and Secondary Markets}
\begin{itemize}
    \item A \textbf{primary market} is a financial market in which new issues of a security, such as a bond or a stock, are sold to initial buyers by the cooperation or government agency borrowing funds.
    \item A \textbf{secondary market} is a financial market in which securities which have been previously issued can be resold.
\end{itemize}

\subsection{Exchanges and Over-the-Counter Markets}
\textit{Secondary markets are organized in two ways:}
\begin{itemize}
    \item \textbf{Exchanges} are where buyers and sellers of securities (or their agents or brokers) meet in one central location to conduct trades, e.g. NYSE, FTSE, Chicago Board of Trade (CBOT).
    \item \textbf{Over-the-Counter (OTC) Markets} are markets in which dealers at different locations have an inventory of securities which they make available for trade (available to be bought and sold).
\end{itemize}

\subsection{Money and Capital Markets}
\begin{itemize}
    \item The \textbf{money market} is a financial market in which only short-term debt instruments (generally with a maturity less than one year) are traded.
    \item The \textbf{capital market} is the market in which longer-term debt (generally with original maturity of one year or greater) and equity instruments are traded.
\end{itemize}

\section{Transaction Costs}

\textit{Transaction costs relate to the time and money spent in carrying out financial transactions.}

\subsection{Transaction costs and Financial Intermediaries}
\begin{itemize}
    \item Financial intermediaries reduce transaction costs because they have developed the expertise and scale to be able to lower them.
    \item This ability to reduce transaction costs allows FIs to be able to provide funds to households and businesses with productive investment opportunities.
    \item Additionally, with low transaction costs, FIs can provide its customers with liquidity services to allow customers to conduct transactions.
\end{itemize}

\section{Risk Sharing}

\textit{An additional benefit of low transaction costs of FIs is that they can help reduce exposure of investors to risk, i.e. uncertainty regarding the returns to assets.}

\subsection{Risk Sharing and Financial Intermediaries}
\begin{itemize}
    \item Financial intermediaries reduce risk through a process known as \textbf{risk sharing}.
    \item FIs create assets with risk characteristics that people are comfortable with, and use this revenue to acquire more risky assets with higher return profiles. This process of risk sharing is also known as \textbf{asset transformation}.
    \item FIs also promote risk sharing by helping individuals to diversify their portfolios.
\end{itemize}

\section{Information Costs / Asymmetric Information}

\textit{In financial markets, one party often does not know enough about the other party to make accurate decisions. This is known as \textbf{asymmetric information}.}

\subsection{Asymmetric Information and Financial Intermediaries}
\begin{itemize}
    \item \textbf{Adverse selection} is the problem created by asymmetric information \emph{before} the transaction occurs. This happens when potential borrowers who may produce a poor (adverse) outcome are the ones who actively seek out loans. Because adverse selection makes it more likely that loans might be made to bad credit risks, lenders may decide not to make any loans even though there are good credit risks in the marketplace.
    \item \textbf{Moral hazard} is the problem created by asymmetric information \emph{after} the transaction occurs. Moral hazard is the risk (hazard) that the borrower might engage in activities that are undesirable (immoral) from the lender’s point of view, because they make it less likely that the loan will be paid back. Because moral hazard lowers the probability that the loan will be repaid, lenders may decide that they would rather not make a loan.
    \item Financial intermediaries can reduce problems arising from asymmetric information by screening potential borrowers (reducing adverse selection) and monitoring borrowers' activities (reducing moral hazard).
\end{itemize}

\section{Financial Market Infrastructure (FMI)}

Supporting the activities of financial markets and financial intermediaries are financial market infrastructures (FMIs). These are primarily payment and settlement system institutions such as real-time gross settlement systems (RTGS), automated clearing houses (ACH) and exchanges.

\subsection{The Caribbean’s Financial Market Infrastructure}
Financial market infrastructures are at varying levels of development across the region. Credit bureau development has been relatively slow.

\section{Types of Financial Institutions}

Financial intermediaries fall into three (3) categories:
\begin{enumerate}
    \item Depository institutions (Banks)
    \item Contractual savings institutions, and
    \item Investment Intermediaries.
\end{enumerate}
These are alternatively termed: (1) Banks, (2) Near-bank financial institutions, and (3) Non-Bank Financial institutions. Examples of these institutions include:
\begin{itemize}
    \item Commercial banks
    \item Savings and loans associations
    \item Credit unions
    \item Insurance companies
    \item Pension funds
    \item Mutual funds
    \item Money market mutual funds
\end{itemize}

\section{Regulation of Financial Institutions}

The financial system remains one of the most heavily regulated sectors in most economies. The government regulates the financial markets and institutions for two (2) main reasons:
\begin{enumerate}
    \item To increase the information available to investors, and
    \item Ensure the soundness of the financial system and by extension the institutions.
\end{enumerate}
Key regulatory concepts and frameworks include:
\begin{itemize}
    \item Basel Committee on Bank Supervision standards (e.g., Basel III)
    \item Solvency I and II (for insurance companies)
    \item PEARLS (for credit unions)
\end{itemize}

\subsection{Micro-prudential Regulation and Supervision}
\begin{itemize}
    \item \textbf{Micro-prudential regulation:} public regulations that aim at maintaining stability of individual institutions.
    \item \textbf{Micro-prudential supervision:} Oversight of specific intermediaries or markets. Examines the responses of an individual intermediary (banks, credit unions, insurance companies, pension funds) or market to exogenous shocks. Neglects the systemic implications of common behavior.
    \item Focuses on individual institutions.
\end{itemize}

\subsection{Macro-prudential Regulation and Supervision}
\begin{itemize}
    \item \textbf{Macro-prudential regulation:} public regulations that aim at maintaining systemic stability.
    \item \textbf{Macro-prudential supervision:} Public oversight that aims at identifying and containing systemic risks. Identification and assessment of risks and vulnerabilities in a whole financial system.
    \item Focuses on systemic risk.
\end{itemize}
