\newpage

\section{Problem Set 2}

\begin{enumerate}

    \item \textbf{Development banks} are financial institutions which typically provide funding
    to small businesses and for other purposes such as education. Consequently,
    they have a comparative advantage in monitoring these types of clients versus
    other institutions such as credit unions which tend to focus primarily on
    households. 
    
    \begin{enumerate}
        \item Using an established theory, explain why development banks may
        have this type of advantage, and why they may be best placed to effect
        transactions with these types of clients.

        \begin{itemize}
            \item Development banks have a comparative advantage in monitoring small businesses mainly due 
                  to their specialization in reducing assymetrical information, a core concept in financial intermediation. The Delegated Monitoring Theory 
                  provides an established framework for understanding this concept. This theory states that financial intermedaries have the expertise and expereience 
                  to efficiently reduce assymetrical information and borrower opportunism, leading to their role as monitors.  
        \end{itemize}
        
        \item Discuss briefly how financial institutions may be losing (gaining) in this advantage 
        given advances in lending approaches.
    \end{enumerate}

    \item \textbf{Transaction Costs Theory and the Efficiency of Financial Markets.} 
    
    \begin{enumerate}
        \item Explain how financial intermediaries reduce transaction costs using the framework
        developed by Benston \& Smith (1976).

        \begin{itemize}
        
        \item Bentson and Smith argue that financial intremedaries were soley created to reduce the transaction costs for the 
              individual lender. This is for two reason, the benefits of their ability to create and own markets and their ability 
              to use economies of scale and economies of scope.

          \item Financial intermediaries create a platform that allows a lender to easily connect with a borrower, this is great for one reason-Costs. The minimum efficent scale of a business
                is higher than that of the individual lender. Before the lender could take a profit from their investment, she/he would be consumed by broker fees, attorney fees, bank fees and the opportunity cost of his/her
                time, making their invesment worthless. Financial intermediaries, on the other hand have lower costs due to economies of scale and economies of scope.  
        \end{itemize}
      
        \item What are the differences between \textbf{economies of scale} and \textbf{economies of scope} 
        in banking? 

        \begin{itemize}
            \item Economies of scale refers to the increase in output due to the increase in the scale of operations. 
            \item Economies of scope refers to the increase in output due to the increase in the instruments or products available.
        \end{itemize}
        
        \item How do they contribute to financial intermediation efficiency? 

            The more of each the easier it is to carry out tasks. 
            
        \item Further, which elements of an institution’s operations are likely to contribute to a 
        divergence from this theory?

        Marginal Product of Capital/Labor. 
    \end{enumerate}

    \item The \textbf{Transaction Cost Theory} (Benston \& Smith, 1976) explains the role of
    financial intermediaries in minimizing the frictions in financial transactions. 
    
    \begin{enumerate}
        \item Explain the concept of \textbf{transaction costs} in financial markets and provide
        examples of different types of transaction costs that arise in direct lending.

        \begin{itemize}
            \item Transactions costs in financial markets operate in the same fashion as it would with Financial Intermediaries. It is the cost of carrying out transactions. Common transaction costs within the financial markets are brokerage fees, liquidity losses, spread fees. 
        \end{itemize}
        
        \item Discuss how financial intermediaries help to reduce transaction costs and
        enhance market efficiency.
    \end{enumerate}

    \item In March 2023, the \textbf{Silicon Valley Bank} (SVB) experienced a bank run in which
    US\$42.0 billion was withdrawn from the institution. To alleviate this bank run,
    the US Federal Reserve responded with prompt regulatory action. 
    
    \begin{enumerate}
        \item Using the \textbf{Diamond-Dybvig Model} (1983), explain the nature and causes of 
        bank runs and the implications for regulation in limiting the broader economic impact 
        of bank runs.
    \end{enumerate}

    \item Explain how \textbf{credit market frictions and external finance premiums} contribute
    to economic activity in the \textbf{Financial Accelerator Theory} (Bernanke, Gertler \&
    Gilchrist, 1999). 
    
    \begin{enumerate}
        \item How do changes in the wealth of households affect borrowing
        costs and financial system stability during economic downturns? 
        
        \item Is there a role for financial system regulation?
    \end{enumerate}

    \item \textbf{Financial repression} is often seen as a major constraint on financial sector
    development, particularly in emerging and developing economies. The
    \textbf{Financial Repression Theory}, as developed by McKinnon \& Shaw (1973),
    argues that government-imposed restrictions on financial markets distort the
    efficient allocation of capital, hinder economic growth, and limit the
    effectiveness of financial intermediation. 
    
    \begin{enumerate}
        \item Explain how this is likely to occur and
        the likely impact upon an economy such as Barbados.
    \end{enumerate}

\end{enumerate}
