
\newcommand{\colorbold}[2]{\textcolor{#1}{\textbf{#2}}}

\newcommand{\itlist}[1]{\begin{itemize} #1 \end{itemize}}
\newcommand{\itenum}[1]{\begin{enumerate} #1 \end{enumerate}}

\newcommand{\al}[1]{\begin{align} #1 \end{align}}
\newcommand{\aln}{\begin{align*}}
\newcommand{\ealn}{\end{align*}}

% Math packages
\usepackage{amsmath, amsfonts, amsthm, amssymb} % AMS packages for math
\usepackage{mathrsfs} % For calligraphic fonts in math
\usepackage{cancel} % For striking out equations
\usepackage{bm} % For bold math symbols

% SI Units
\usepackage{siunitx} % For SI units
\sisetup{locale = FR} % Setup for SI units localization

% Graphics and Figures
\usepackage{graphicx} % For including graphics
\usepackage{float} % For figure placement control
\usepackage{import} % For importing graphics
\usepackage{pdfpages} % For including PDF pages
\usepackage{transparent} % For transparency support
\usepackage{pdflscape} % Optional for landscape pages

% Theorem styles
\usepackage{thmtools} % For defining theorem styles
\usepackage[framemethod=TikZ]{mdframed} % For framed theorems

% Define theorem styles
% theorems
\usepackage{thmtools}
\usepackage[framemethod=TikZ]{mdframed}
\mdfsetup{skipabove=1em,skipbelow=0em, innertopmargin=5pt, innerbottommargin=6pt}


\theoremstyle{definition}

\makeatletter


\@ifclasswith{report}{nocolor}{
    \declaretheoremstyle[headfont=\bfseries\sffamily, bodyfont=\normalfont, mdframed={ nobreak } ]{thmredbox}
    \declaretheoremstyle[headfont=\bfseries\sffamily, bodyfont=\normalfont]{thmbluebox}
    \declaretheoremstyle[headfont=\bfseries\sffamily, bodyfont=\normalfont]{thmblueline}
    \declaretheoremstyle[headfont=\bfseries\sffamily, bodyfont=\normalfont, numbered=no, mdframed={ rightline=false, topline=false, bottomline=false, }, qed=\qedsymbol ]{thmproofbox}
    \declaretheoremstyle[headfont=\bfseries\sffamily, bodyfont=\normalfont, numbered=no, mdframed={ nobreak, rightline=false, topline=false, bottomline=false } ]{thmexplanationbox}
    \AtEndEnvironment{eg}{\null\hfill$\diamond$}%
}{
    \declaretheoremstyle[
    headfont=\color{black},  % Fake bold for Concrete Math
    bodyfont=\normalfont,
    mdframed={
        linewidth=1pt,  % Set linewidth to 0 to remove the lines
        rightline=false, topline=false, bottomline=false,
        linecolor=myorange,  % Change line color to black (will have no effect since linewidth is 0)
        backgroundcolor=myorange!5,  % Change background color to white (no color)
    }
    ]{thmorangebox} 

    \declaretheoremstyle[
        headfont=\color{black},  % Fake bold for Concrete Math
        bodyfont=\normalfont,
        mdframed={
            linewidth=1pt,  % Set linewidth to 0 to remove the lines
            rightline=false, topline=false, bottomline=false,
            linecolor=brown,  % Change line color to black (will have no effect since linewidth is 0)
            backgroundcolor=brown!5,  % Change background color to white (no color)
    }
    ]{thmbrownbox} 

    \declaretheoremstyle[
        headfont=\bfseries\sffamily\color{NavyBlue!70!black}, bodyfont=\normalfont,
        mdframed={
            linewidth=2pt,
            rightline=false, topline=false, bottomline=false,
            linecolor=NavyBlue, backgroundcolor=NavyBlue!5,
        }
    ]{thmbluebox}

    \declaretheoremstyle[
        headfont=\bfseries\sffamily\color{NavyBlue!70!black}, bodyfont=\normalfont,
        mdframed={
            linewidth=2pt,
            rightline=false, topline=false, bottomline=false,
            linecolor=NavyBlue
        }
    ]{thmblueline}

    \declaretheoremstyle[
        headfont=\bfseries\sffamily\color{RawSienna!70!black}, bodyfont=\normalfont,
        mdframed={
            linewidth=2pt,
            rightline=false, topline=false, bottomline=false,
            linecolor=RawSienna, backgroundcolor=RawSienna!5,
        }
    ]{thmredbox}

    \declaretheoremstyle[
        headfont=\bfseries\sffamily\color{RawSienna!70!black}, bodyfont=\normalfont,
        numbered=no,
        mdframed={
            linewidth=2pt,
            rightline=false, topline=false, bottomline=false,
            linecolor=RawSienna, backgroundcolor=RawSienna!1,
        },
        qed=\qedsymbol
    ]{thmproofbox}

    \declaretheoremstyle[
        headfont=\bfseries\sffamily\color{NavyBlue!70!black}, bodyfont=\normalfont,
        numbered=no,
        mdframed={
            linewidth=2pt,
            rightline=false, topline=false, bottomline=false,
            linecolor=NavyBlue, backgroundcolor=NavyBlue!1,
        },
    ]{thmexplanationbox}
}





\declaretheorem[style=thmorangebox, name=Definition]{definition}
\declaretheorem[style=thmbrownbox, numbered=no, name = {}]{A}
\declaretheorem[style=thmorangebox, numbered=no, name=Question]{Q}
\declaretheorem[style=thmredbox, name=Proposition]{prop}
\declaretheorem[style=thmredbox, name=Theorem]{theorem}

\@ifclasswith{report}{nocolor}{
    \declaretheorem[style=thmproofbox, name=Proof]{replacementproof}
    \declaretheorem[style=thmexplanationbox, name=Proof]{explanation}
    \renewenvironment{proof}[1][\proofname]{\begin{replacementproof}}{\end{replacementproof}}
}{
    \declaretheorem[style=thmproofbox, name=Proof]{replacementproof}
    \renewenvironment{proof}[1][\proofname]{\vspace{-10pt}\begin{replacementproof}}{\end{replacementproof}}

    \declaretheorem[style=thmexplanationbox, name=Proof]{tmpexplanation}
    \newenvironment{explanation}[1][]{\vspace{-10pt}\begin{tmpexplanation}}{\end{tmpexplanation}}
}

\makeatother


\declaretheorem[style=thmblueline, numbered=no, name=Remark]{remark}
\declaretheorem[style=thmblueline, numbered=no, name=Note]{note}

\newtheorem*{uovt}{UOVT}
\newtheorem*{notation}{Notation}
\newtheorem*{previouslyseen}{As previously seen}
\newtheorem*{problem}{Problem}
\newtheorem*{observe}{Observe}
\newtheorem*{property}{Property}
\newtheorem*{intuition}{Intuition}


\usepackage{etoolbox}
\AtEndEnvironment{vb}{\null\hfill$\diamond$}%
\AtEndEnvironment{intermezzo}{\null\hfill$\diamond$}%
% \AtEndEnvironment{opmerking}{\null\hfill$\diamond$}%

% http://tex.stackexchange.com/questions/22119/how-can-i-change-the-spacing-before-theorems-with-amsthm
% \def\thm@space@setup{%
%   \thm@preskip=\parskip \thm@postskip=0pt
% }

\newcommand{\oefening}[1]{%
    \def\@oefening{#1}%
    \subsection*{Oefening #1}
}

\newcommand{\suboefening}[1]{%
    \subsubsection*{Oefening \@oefening.#1}
}

\newcommand{\exercise}[1]{%
    \def\@exercise{#1}%
    \subsection*{Exercise #1}
}

\newcommand{\subexercise}[1]{%
    \subsubsection*{Exercise \@exercise.#1}
}


\usepackage{xifthen}

\def\testdateparts#1{\dateparts#1\relax}
\def\dateparts#1 #2 #3 #4 #5\relax{
    \marginpar{\small\textsf{\mbox{#1 #2 #3 #5}}}
}

\def\@lesson{}%
\newcommand{\lesson}[3]{
    \ifthenelse{\isempty{#3}}{%
        \def\@lesson{Lecture #1}%
    }{%
        \def\@lesson{Lecture #1: #3}%
    }%
    \subsection*{\@lesson}
    \testdateparts{#2}
}

% \renewcommand\date[1]{\marginpar{#1}}
% Custom commands
\newcommand\N{\ensuremath{\mathbb{N}}} % Natural numbers
\newcommand\R{\ensuremath{\mathbb{R}}} % Real numbers
\newcommand\Z{\ensuremath{\mathbb{Z}}} % Integers
\newcommand\C{\ensuremath{\mathbb{C}}} % Complex numbers
\newcommand\hr{ % Horizontal rule
    \noindent\rule[0.5ex]{\linewidth}{0.5pt}
}

% Notes and comments
\usepackage{tcolorbox} % For colored boxes
\tcbuselibrary{breakable}

\newenvironment{verbetering}{%
    \begin{tcolorbox}[colback=white, colframe=green!60!black, title=Opmerking, fonttitle=\sffamily, breakable]
}{%
    \end{tcolorbox}
}

% Fancy headers and footers
\usepackage{fancyhdr}  % For fancy headers
\pagestyle{fancy}  % Enable fancy page style
\setlength{\headsep}{14pt}
\setlength{\headheight}{15pt} % Adjust based on your header content
\fancyhf{}  % Clear all header and footer fields
\fancyfoot[R]{\thepage} % Footer: Page number on the right
% Left header: Current section title
\fancyhead[L]{\leftmark}
% Right header: Line at the top
\renewcommand{\headrulewidth}{0pt}  % Remove the header line


% Redefine the headrule to draw a line under the header
\renewcommand{\headrule}{%
    \vspace{2pt} % Add a small space above the line
    \hrule width \headwidth height 0.4pt % Draw the line
    \vspace{2pt} % Add a small space below the line
}

% Ensure all header text fits within page margins
\newcommand{\setheader}[1]{%
    \fancyhead[R]{\small #1 | \thepage}  % Right header with title and page number
}




% Figure support
\newcommand{\incfig}[1]{%
    \def\svgwidth{\columnwidth}
    \import{./figures/}{#1.pdf_tex}
}


\usepackage{enumerate}
\usepackage{tikz}
\usepackage{multicol}
\usepackage{booktabs}
