
\subsection*{Macaronia's Liquidity and Capital Ratios}

Nominally, Macaronia's liquidity and capital adequacy ratios are both above the set prudential benchmarks. Over the years, banks and credit unions have attained liquidity ratios of \textbf{75\%} (a \textbf{25\% increase}) and \textbf{78.8\%} (a \textbf{26.3\% increase}), respectively. Similarly, capital adequacy ratios have reached \textbf{14.8\%} (a \textbf{4.8\% increase}) for banks and \textbf{12.8\%} (a \textbf{4.8\% increase}) for credit unions. On paper, these ratios appear strong, suggesting that financial institutions are solvent and have sufficient liquidity to meet short-term obligations.

\subsubsection{Credit Growth and Its Impact}
However, the effectiveness of these ratios is being undermined by the rapid expansion of credit. While the ratios are rising, the scale of credit being extended to Macaronia's households has grown at a significant rate, rendering the ratios less meaningful in practice. The credit extended by banks grew from 1.5\% in 2015 to 8.8\% in 2021, while credit extended by credit unions grew from 2.3\% in 2015 to 13.3\% in 2021. 

\subsubsection{Exposure to Real Estate Investment Trusts (REITs)}
Both banks and credit unions are heavily exposed to Real Estate Investment Trusts (REITs), with banks having 15\% of their assets tied to REITs and credit unions having 25\%. REITs are highly volatile, with their values subject to significant fluctuations based on market conditions, interest rates, and economic cycles. In times of economic uncertainty, these assets put financial institutions in a precarious position.

\subsubsection{Impact of High Public Debt}
The public sector's debt level, which stands at 80\% of GDP, raises serious concerns about the government's capacity to meet its own financial obligations, let alone provide bailouts to struggling financial institutions in the event of a crisis.

\subsubsection{Adjusted Liquidity and Capital Adequacy Ratios}
When nominal ratios are adjusted for credit extension, risk exposure, and REIT exposure, the real value is significantly lower. This adjustment uncovers underlying vulnerabilities, emphasizing the need for a more accurate assessment of solvency and liquidity risks. 

\[
\text{Adjusted Liquidity Ratio for Banks} = \frac{75.0}{1.878} \approx 39.94, \quad \text{Adjusted Solvency Ratio for Banks} = \frac{14.8}{1.878} \approx 7.88
\]

\[
\text{Adjusted Liquidity Ratio for Credit Unions} = \frac{78.8}{2.122} \approx 37.14, \quad \text{Adjusted Solvency Ratio for Credit Unions} = \frac{12.8}{2.122} \approx 6.03
\]

\subsection{External Factors Triggering the Crisis}

\subsubsection{Global Financial Turmoil}
The failure of a major U.S. financial institution in 2021 sent shockwaves through global credit markets, tightening financial conditions worldwide and drastically reducing investor confidence. Macaronia, with its deep financial ties to the international system, felt the immediate effects.

\subsubsection{Impact on Macaronia’s Stock Market and Reserves}
Macaronia’s stock market, which had cross-listed securities with the U.S., saw a sharp decline as investors fled. Concurrently, international reserves began depleting rapidly as foreign investors withdrew their capital, worsening the country’s financial position and putting downward pressure on the currency.

\subsubsection{Liquidity Crisis in the Banking Sector}
The outflow of foreign capital and declining investor confidence led to a massive withdrawal of funds by depositors and business owners. Many financial institutions, already weakened by excessive credit expansion and high-risk exposure to REITs, struggled to meet these withdrawals. As a result, two commercial banks failed outright, exacerbating the financial crisis.

\subsubsection{International Monetary Fund (IMF) Intervention}
The IMF intervened by offering financial support to stabilize Macaronia's reserves. However, this aid raised concerns about long-term debt sustainability and the economic restrictions that would likely accompany it. Investors saw the reliance on external aid as a signal of deeper structural weaknesses, further undermining confidence in the economy.

\subsection{Conclusion: A Fragile Financial System}

The combination of overexposure to volatile REITs, aggressive lending, the precarious state of public finances, and the realization of insufficient real liquidity and solvency ratios led to an extremely fragile financial system. A simple trigger, in the form of the U.S. financial institution's failure, sent Macaronia into crisis. This underlines the vulnerability of the country’s financial system to external shocks and the need for more robust financial practices and better economic diversification.
