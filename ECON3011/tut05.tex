\newpage

\section{Problem Set 5}



\begin{enumerate}

    \item \textbf{Open Market Operations (OMOs)} are the primary tool for adjusting the
    federal funds rate and influencing liquidity in the banking system. Using
    graphs where appropriate:
    
    \begin{enumerate}
        \item Explain in detail how an open market purchase affects:
        \begin{enumerate}
            \item The supply of reserves,
            \item The federal funds rate, and
            \item The broader economic implications on lending and inflation.
        \end{enumerate}

        \item Consider a scenario where the economy is experiencing rising inflation and
        excessive credit expansion. Describe how the Federal Reserve would use an
        open market sale to:
        \begin{enumerate}
            \item Reduce excess liquidity,
            \item Increase interest rates, and
            \item Curb inflationary pressures.
        \end{enumerate}


        \item Under what conditions might an open market operation fail to influence the
        federal funds rate? Discuss the role of:
        \begin{enumerate}
            \item The interest rate on excess reserves (IER) as a policy tool,
            \item The liquidity trap, and
            \item The scenario where the supply curve intersects the flat portion of
            the demand curve.
        \end{enumerate}
    \end{enumerate}

    \item \textbf{Discount lending} serves as a critical monetary policy and financial stability
    tool during periods of liquidity stress.

    \begin{enumerate}
        \item Define and explain the three types of discount lending offered by the
        Federal Reserve and discuss the specific economic conditions under
        which each type can be used.

        \item Suppose that the demand for reserves intersects the vertical section of the
        supply curve, meaning no banks are borrowing from the Fed.
        \begin{enumerate}
            \item If the Fed lowers the discount rate, explain why this action would
            have no immediate impact on the federal funds rate.
            \item Discuss the rationale for setting the discount rate above the federal
            funds rate target in normal economic conditions.
        \end{enumerate}

        \item Under what circumstances would a reduction in the discount rate lead to
        a direct and significant impact on the federal funds rate? Consider:
        \begin{enumerate}
            \item The role of borrowed reserves (BR) in the supply of reserves,
            \item A financial crisis scenario where banks become reliant on discount
            window borrowing, and
            \item The potential for moral hazard when banks excessively use discount
            lending.
        \end{enumerate}
    \end{enumerate}

    \item \textbf{The reserve requirement ratio} affects both money creation and liquidity in the
    banking system.

    \begin{enumerate}
        \item Explain the impact of a higher reserve requirement ratio on:
        \begin{enumerate}
            \item The demand for reserves,
            \item The federal funds rate, and
            \item The money multiplier effect on the economy.
        \end{enumerate}

        \item Compare the effectiveness of reserve requirement changes to open
        market operations in controlling short-term interest rates.
        \begin{enumerate}
            \item Why are OMOs preferred for fine-tuning monetary policy?
            \item What are the downsides of adjusting reserve requirements
            frequently?
            \item How do reserve requirements affect bank profitability and credit
            availability?
        \end{enumerate}
    \end{enumerate}

    \item \textbf{During financial crises}, traditional monetary policy tools are often insufficient,
    requiring the use of unconventional interventions.

    \begin{enumerate}
        \item Explain how the Federal Reserve responded to the 2008 financial
        crisis by implementing new liquidity facilities such as the Term Auction
        Facility (TAF), the Primary Dealer Credit Facility (PDCF), and the
        Term Securities Lending Facility (TSLF) to help stabilize financial
        markets.

        \item The lender of last resort (LOLR) function can introduce moral hazard.
        \begin{enumerate}
            \item Define moral hazard in the context of central banking.
            \item How might banks take excessive risks if they expect continuous
            central bank support?
            \item What regulatory measures can be used to reduce moral hazard
            while ensuring financial stability?
        \end{enumerate}

        \item During financial crises, central banks often expand the range of
        eligible collateral for lending operations.
        \begin{enumerate}
            \item Why is this necessary?
            \item What types of assets become eligible as collateral during crises?
            \item What are the potential risks of this approach for the central
            bank’s balance sheet?
        \end{enumerate}
    \end{enumerate}

    \item \textbf{Quantitative Easing (QE)} is a non-traditional monetary policy tool used when
    short-term interest rates approach zero.

    \begin{enumerate}
        \item Compare Quantitative Easing (QE) to Open Market Operations
        (OMOs) by addressing:
        \begin{enumerate}
            \item The differences in scale and objectives,
            \item How QE targets long-term interest rates, and
            \item Why QE is used when interest rates are already near zero.
        \end{enumerate}

        \item What economic conditions make QE necessary? Discuss:
        \begin{enumerate}
            \item Deflationary risks,
            \item Liquidity traps, and
            \item The need to stimulate lending and economic activity when
            traditional rate cuts become ineffective.
        \end{enumerate}

        \item Discuss one major long-term risk of QE, particularly regarding:
        \begin{enumerate}
            \item The potential for asset bubbles,
            \item The impact on wealth inequality, and
            \item The challenge of unwinding QE without disrupting financial
            markets.
        \end{enumerate}
    \end{enumerate}

    \item \textbf{When traditional monetary policy loses effectiveness}, central banks consider
    unconventional tools like Negative Interest Rate Policy (NIRP).

    \begin{enumerate}
        \item Define a liquidity trap and explain why:
        \begin{enumerate}
            \item Traditional interest rate cuts may become ineffective,
            \item The public and businesses prefer to hold cash instead of
            investing, and
            \item Inflation expectations become anchored at low levels, making
            monetary stimulus ineffective.
        \end{enumerate}

        \item Explain how Negative Interest Rate Policy (NIRP) is intended to
        stimulate the economy. Discuss:
        \begin{enumerate}
            \item How NIRP affects commercial banks,
            \item The incentive for banks to lend instead of hoarding excess
            reserves, and
            \item The impact on exchange rates and global capital flows.
        \end{enumerate}
    \end{enumerate}

\end{enumerate}

\subsection{Question 2 Solution}

The Federal Reserves sell government securties(commonly treasury bonds) to the commercial banks(or any relevant/related financial institutions). 
The banks pay for these transactions by transferring funds from their reserve account. This directly reduces reserves from the banking system. Since 
reserves form the base money needed to create loans or meet short-term obligations, reducing them overall drains liquidity from the banking system. 

When the FED drains reserves, banks have less money to lend to each other in the Federal funds market. To meet their reserve requirement or fuel funding needs, banks 
needing to borrow reserves must now compete more intensely for scarcer funds, bidding up the interest they are willing to pay. The increase in the federal funds rate 
typically ripples through the financial system. Banks will increase the rates they charge their customer for loans (like the prime rate), and other short-term and even
longer term interest tend full suit as the overall cost of borrowing increases.

As explained in the previous text, the open market sale leads to higher interest rates across the economy. Higher interest rates make borrowing more expensive for businesses and households. 
This leads to a decrease in consumption and investment spending as it is more expensive to take out loans to purchase big-ticket items like homes, appliances, cars or finance new projects, equipment purchasees 
or inventory expansion. The reduction in consumption and investment decreases the overall supply and demand of goods and services within the economy. Bussinesses will now face less pressure to raise prices. In fact, 
they may need to compete on prices to attract customers. This slowdown in price increases helps curb inflation. 




