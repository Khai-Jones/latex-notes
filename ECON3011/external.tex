
\section{External Issues: Amplifyng the Crisis}

\subsection{ Collapse of a Major U.S. Financial Institution}


The Globalization of financial systems means that Macaronia’s economy is not immune to such external shocks. The moderate correlation between global economic growth and Macaronia’s economic growth (0.44) indicates that Macaronia's economy is somewhat exposed to global trends. Cross-listing of securities means that Macaronia's stock market is suffering from U.S. market volatility. The global credit crunch has limited foreign funding, worsening capital flight and currency depreciation, further destabilizing the economy.
For instance, the collapse has led to a tightening of global credit conditions, making it more difficult for Macaronia’s banks to access foreign funding. This has exacerbated liquidity shortages and increased borrowing costs, further straining the financial system.


\subsection{Global Economic Contractions}
Global economic contractions have further strained Macaronia's economy, although domestic factors remain the primary drivers of the crisis. While global trends affect Macaronia, as shown by the correlation between Macaronia's economic growth and global economic growth (0.44), domestic credit expansion has played a larger role in the current crisis. The correlation between credit union lending and economic growth (0.77) highlights that Macaronia's economic expansion has been debt-driven rather than export-driven. Slower global growth is dampening inflationary pressures, as indicated by the negative correlation between inflation and global growth (-0.46), but this provides little relief given the domestic challenges.
For example, the slowdown in global trade has reduced demand for Macaronia’s exports, further weakening the economy. This has been compounded by declining remittances from overseas workers, which have traditionally been a key source of foreign exchange.
\subsection{Capital Flight}

Capital flight has accelerated in response to the economic crisis, driven by concerns over currency depreciation and financial instability. The correlation between public debt to GDP and exchange rate (-0.68) confirms that capital flight is accelerating currency depreciation. Capital flight is also reducing financial stability, as shown by the correlation between liquidity ratios and exchange rate (-0.28), creating a vicious cycle of economic decline. Foreign investors are withdrawing from Macaronia's markets, further destabilizing the economy, as evidenced by the stock market decline and capital flight.
For instance, the withdrawal of foreign capital has led to a sharp decline in asset prices, particularly in the real estate and stock markets. This has further eroded investor confidence and increased the risk of a full-blown financial crisis.

\subsection{ Decline in Foreign Direct Investment (FDI)}
The decline in FDI is a significant concern, as it reflects eroding investor confidence in Macaronia's economic prospects. Investors fear further currency depreciation, which would erode returns on their investments, as indicated by the expected correlation between exchange rate and FDI (-0.68). Falling equity markets signal higher investment risks, discouraging FDI inflows. High public debt levels make Macaronia less attractive to foreign investors, as they perceive higher sovereign risk, as shown by the correlation between public debt to GDP and FDI (-0.45). The collapse of the real estate market, a key driver of FDI, has further reduced investor confidence.
For example, the cancellation of several major infrastructure projects due to funding shortages has sent a negative signal to foreign investors, further reducing FDI inflows.

