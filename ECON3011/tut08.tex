\newpage

\section{Problem Set 8}

1. Define physical and transition risks in the context of climate-related financial
risks. Provide one example of each and explain how they can affect financial
institutions.

2. Explain the transmission channels through which climate-related risks can
impact financial stability. Highlight at least three distinct mechanisms.

3. Discuss the concept of double materiality as it applies to sustainable finance.
How does it differ from traditional financial materiality?

4. Describe the role of climate stress testing in assessing financial system
resilience. How does it differ from traditional macro-financial stress tests?

5. Identify two tools or strategies central banks or regulators can use to address
climate-related risks in the financial system. Briefly explain their purpose and
potential limitations.

6. How can sustainable finance contribute to financial stability and the transition
to a low-carbon economy? Provide one example of a sustainable finance
instrument and its potential impact.
