\newpage

\section{Problem Set 1}

\begin{enumerate}

    \item Define \textbf{financial systems}. What are their primary components, and why are
    they significant in the economy?
       \begin{itemize}
           \item Financial systems are a network of instruments, institutions, markets and regulation that facilitate the flow of funds within the economy. It comprises of banks, near-bank institutions(such as credit unions), non-bank financial institutions (such as insurance companines), etc. These systems are detrimental because 
                 they efficently help allocate resources efficently by channelling funds from savers (those who have surplus funds) to borrowers (those in need of funds). For example, households may need financing for big purchases like cars or houses, while businesses may seek capital to expand their operations through investments in
                 infrastructure or equipment. 
       \end{itemize}

    \item What is the difference between \textbf{direct and indirect finance}? Provide examples
    of each.
      \begin{itemize}
          \item Direct finance occurs when borrower obtain funds directly from lenders without the use of financial intermediaries. This typically occurs through financial markets, such as when instruments (corporation bonds or stocks) are bought by investors. Indirect finance, on the other hand involves financial intermediation-such as banks 
                or credit unions who channel funds from savers to borrowers. For instance, when an individual deposits money into a bank, the bank may lend a portion of those funds to businesses or households, effectively transfering funds indirectly.
      \end{itemize}
    \item Explain the importance of \textbf{financial intermediaries} in reducing transaction
    costs and risks.
      \begin{itemize}
          \item Minimum efficent scale is larger for businesses than most individuals can invest. Someone with \$100, \$1000, \$1000, \$10,000 or even \$100,000 to invest would 
                have a hard time turning over a profit. This in because most of their profits would be eaten up by transaction costs like banking and broker fees, dealer spreads, attorney 
                fees, liquidity and diversification losses and the opportunity cost of his or her time. Financial intermediaries on the other hand are specialized to deal with transaction costs due to economics of scale 
                and risk sharing. By pooling together funds from many investors, they also enable diversification and risk-sharing, making investments more efficient and less risky for individual particpants.  
      \end{itemize}

    \item Discuss the function of \textbf{financial markets}. How do they facilitate the connection
    between savers and borrowers?
       \begin{itemize}
           \item Financial markets serve the primary function of channeling funds from savers to those who need it for either consumption or investment purchases. Savers provide funds by purchasing financial instruments 
                 such as bonds, stocks or deposits. These instruments represent a financial claim or future income. Borrowers such as businesses or governments seeking funding for public projects-issue these instruments to raise capital. 
       \end{itemize}

    \item Differentiate between \textbf{primary and secondary markets}. How do they contribute
    to the overall financial system?

    \item Explain the key differences between \textbf{debt markets and equity markets}.

    \item Describe the structure of the \textbf{Caribbean financial system}. What are the
    primary institutions involved?

    \item Discuss the \textbf{challenges facing the development} of financial market
    infrastructure in the Caribbean.

    \item Why is the \textbf{financial sector heavily regulated}? Outline the two main reasons for
    regulation.

    \item Explain how \textbf{financial intermediaries manage risks} through risk sharing and
    asset transformation.

\end{enumerate}
