\newpage

\section{Problem Set 4}

\begin{enumerate}

    \item The \textbf{2007–2009 Global Financial Crisis (GFC)} highlighted the interconnectedness
    of financial institutions and systemic risk transmission. The crisis led to an
    overhaul of monetary policy strategies and regulatory frameworks globally.
    
    \begin{enumerate}
        \item How did central banks, including the Federal Reserve and the European Central
        Bank (ECB), use unconventional monetary policies (e.g., quantitative easing,
        forward guidance) to mitigate the impact of the crisis?

        \begin{itemize}
            \item The FED and the ECB implemented two new policies, quantitative easing and forward guidance. T
        \end{itemize}

        
        \item Compare these responses to the role of Caribbean central banks, such as the
        Eastern Caribbean Central Bank (ECCB) and the Bank of Jamaica (BOJ), in
        managing financial shocks.
    \end{enumerate}

    \item \textbf{Bank runs} have historically destabilized financial systems, with depositors
    withdrawing funds due to panic, loss of confidence, or perceived insolvency risks.
    The collapse of Silicon Valley Bank (SVB) in March 2023 prompted significant
    central bank intervention.
    
    \begin{enumerate}
        \item Using the SVB collapse as a case study, discuss how liquidity risk and interest
        rate risk exposure contributed to the failure of the bank.

        \begin{itemize}
            \item The Liquidity Theory, particularly the diamond-dybvig model, highligts the fragile tension 
                  between the equilibrium of bank-runs and optimal efficient risk-sharing. A notable chartaceristic 
                  of bank runs is that they are a self-fulfilling prophecy, as even the mere mention or expectation of one incites 
                  panic and thus it's occurence. This implies that a bank can be completely sound and still fail due to a bank run. The Silicone 
                  Valley Bank held an undiversified portfolio with a significant portion contributing to a single sector (tech and venture), mortgage-backed securties 
                  and treasury-bonds. When tech and venture startups faced difficulties in securing and raising capital with the bank, they began withdrawing their deposits 
                  from SVB. Eventually word of their financial struggle broke through to the media, propogating panic and leading depositors to commence a mass withdrawl. Unable to liquidate 
                  their long-term assets to meet their short-term obligations the Silicon Valley Bank collapsed. In tandeem with liquidity risks the bank was also susceptible to interest rate risks-the flutuctions of 
                  interest rates due to financial market, policy implementation or economic conditions-which mainly affected their investments in treasury bonds and mortgaged-backed securties. At the time the US economy was experiencing 
                  inflation. The FED rose interest rates to discourage borrowing and expenditure but indirectly worsened SVB position. Given their investments long-term assets, higher interest rates depreciated their value making it difficult 
                  for the bank to meet it's obligations. 
        \end{itemize}
        \item Analyse the role of deposit insurance schemes and LOLR mechanisms in
        preventing crises in banking systems.
        
        \item Should Caribbean central banks expand their LOLR facilities to support
        nonbank financial institutions (NBFIs) (e.g., credit unions and insurance
        companies)? Give your perspective on this matter.
    \end{enumerate}

    \item \textbf{Monetary policy effectiveness} is often tied to central bank independence,
    credibility, and policy transparency. Caribbean economies have unique monetary
    policy frameworks, ranging from inflation targeting (BOJ) to fixed exchange rate
    regimes (ECCB).
    
    \begin{enumerate}
        \item Compare the degree of independence of the Federal Reserve, European Central
        Bank (ECB), and Eastern Caribbean Central Bank (ECCB). How does
        institutional design affect policy outcomes?
        
        \item Discuss the trade-offs between inflation targeting and exchange rate
        management in small open economies, using the Bank of Jamaica (BOJ) and
        Central Bank of Barbados (CBB) as examples.
        
        \item Given high global inflation rates, should Caribbean central banks prioritize
        inflation control over economic growth? Support your argument with real-world
        cases.
    \end{enumerate}

\end{enumerate}
