\newpage

\section*{Question 1} 

Explain how the governance structure of credit unions differs from that of commercial banks and dicuss the implications for credit allocation. Given the 
"common bond" membership requirement of credit unions, discuss whether credit unions are better suited than commercial banks to address financial exclusion in rural 
and undeserved communities. 


\subsection*{Solution:}


The governance structure of a credit union significantly diverges from that of commercial banks in three principle areas: member ownership and democratic control, regulatory frameworks, and 
non-profit operational models.

Firstly, credit unions are member-owned institutions, where customers are also stakeholders. Credit unions have a "one member, one vote" principle, ensuring that each member possesses an equal voice 
in the decision, irrespective of their deposit size. Conversely, commercial banks are typically structured with shareholder influence proportional to indiviual shareholdings, where larger shares equate to 
greater control. 

A credit unions governance structure differ from that of commercial banks in three major areas, member-owned (democratic governance), regulaions and it's non-profit structure. 
Credit Unions are supported by their members who are also their customers, it operates on a one member one vote basis. This means that regardless of the size of their deposits, 
every member has a vote. Central Banks are influnced depending on the percentage share of it's individual shareholders. The more shares held the more power over the bank the share holder has. A credit union's membership 
is also based on the a "common/shared bond" which is based on geographical location, Work Affiliation, etc. 


\section*{Question 2}

Discuss three (3) major barriers to financial access in Caribbean economies. Propose two policy interventions that could help overcome these barriers 
and expand financial access in the Caribbean.

\subsection*{Solution: }

The Caribbean Region is impended by various barriers: Collateral Requirements, Borrowing Costs, Contractual Frameworks, etc. Large proportions of loans require high collateral, often exceeding 80\% in some countries, compared to 38\% in Latin America. High intrest rates 
and wide spreads between deposit and loan rates are common in some countries. Many Caribbean countries rank poorly in contract enforcement and property registration. There are a lack of integrated legal frameworks for secured transcations, collateral registries, or legl priority for secured credutiors in bankruptcy. 


\section*{Question 3}

Discuss the role of insurance companies in financial intermediation, particularly in
risk pooling and redistribution. How can insurance companies contribute to
macroeconomic stability, particularly in mitigating economic shocks and
smoothing consumption.

\subsection*{Solution: }


\section*{Question 4}

Insurance penetration remains low in many developing economies. What policy
measures can governments implement to increase the role of insurance in financial
intermediation?

\subsection*{Solution}

To enhance the role of insurance's financial intremediation, governments can implement a variety of 
policy measures. Raising public awareness of the benefits of insurance and increasing insurance literacy should be the governments first 
steps. National campaigns can educate individuals and businesses on the importance of risk management and the various types of coverage available. 
Integrating insurance topics into financial literacy education programs can also raise demand for insurance products. 


\section*{Question 5}

Define adverse selection and moral hazard in the context of insurance markets,
providing real-world examples. Explain how insurers may mitigate these risks
through screening mechanisms, deductibles, and risk-based pricing.


\section*{Question 6}

Using examples from other developing countries, propose strategies for improving
capital market depth and liquidity in the Caribbean.


