\newpage

\section{Problem Set 9}

\begin{enumerate}

    \item 
    \textbf{A.} Define adverse selection and moral hazard, and explain how asymmetric information contributed to the 2008 financial crisis. 
    
    \textbf{B.} How did the Dodd-Frank Act attempt to address these issues in the U.S. regulatory framework?

    \item
    \textbf{A.} Explain the economic rationale for deposit insurance. 
    
    \textbf{B.} Discuss the moral hazard problem it creates and how supervisory mechanisms such as risk-based premiums help mitigate it. Refer to examples in the U.S. and the ECCU.

    \item
    \textbf{A.} Compare and contrast the roles of the Federal Reserve, the FDIC, and the OCC in the supervision of financial institutions. Highlight their functions during crises and how they contribute to financial stability.

    \item
    \textbf{A.} Summarize the three pillars of the Basel Framework.
    
    \textbf{B.} Explain how Pillar 2 enhances supervisory oversight, especially in the context of emerging financial innovations.

    \item
    \textbf{A.} Describe how rapid financial innovation (e.g., cryptocurrencies, digital banking) poses new challenges to traditional regulatory frameworks.
    
    \textbf{B.} What supervisory strategies are being explored to address these challenges?

    \item
    \textbf{A.} Explain the rationale for harmonizing deposit insurance systems across CARICOM.

\end{enumerate}
