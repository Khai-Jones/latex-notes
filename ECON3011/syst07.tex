\newpage

\lecturetitle{7}{Systemic Risk and Financial Stability}

\subsection{Definition and Importance}
Systemic risk refers to the risk that disruptions affecting a part of the financial system (e.g., a single institution or market) can trigger widespread instability or a cascade of failures, potentially leading to the collapse or severe impairment of the entire financial system. Understanding and mitigating systemic risk is crucial for maintaining financial stability, which is essential for sustainable economic growth.

\subsection{Key Dimensions}
Systemic risk has distinct characteristics:
\begin{itemize}
    \item \textbf{Time-Series Dimension:} Systemic risk evolves over the financial cycle. It tends to build up endogenously during periods of economic expansion and credit booms, leading to increased financial fragility. This fragility makes the system vulnerable, and crises often occur in waves following the boom phase, with risk decreasing during recessions as imbalances unwind (often painfully).
    \item \textbf{Cross-Sectional Dimension:} Not all institutions contribute equally to systemic risk. Some are more systemically important due to their size, complexity, interconnectedness, and substitutability. Measuring this dimension helps identify which intermediaries pose the greatest risk to the system.
\end{itemize}

\subsection{Endogenous Nature and Shock Transmission}
A key aspect of systemic risk is its \textbf{endogenous nature}. It is not just an external force but can be actively taken on by financial intermediaries through their collective behavior (e.g., correlated risk-taking during asset price bubbles like real estate, excessive leverage). This endogenous build-up can create shocks from within the financial sector itself.

Systemic risk also involves the \textbf{transmission and amplification of shocks}. An initial shock (which could be idiosyncratic, affecting one firm, or endogenous, stemming from built-up imbalances) can spread throughout the system via various contagion channels, magnifying its impact. It's important to distinguish between the initial build-up of vulnerabilities (endogenous risk creation) and the subsequent propagation of distress (contagion).

\section{Drivers and Amplifiers of Systemic Risk}

Several factors contribute to the build-up and amplification of systemic risk.

\subsection{Financial Imbalances}
Historical analysis across many countries shows that periods of excessive credit growth are often the best leading indicators of impending financial instability. This links closely with asset price bubbles (particularly in real estate and equity markets) and the build-up of leverage within the financial system. These imbalances create financial fragility.

\subsection{Wholesale Markets and Interlinkages}
The growing importance of wholesale funding markets and the 'shadow banking' system has introduced new sources of systemic risk. Interlinkages have increased significantly:
\begin{itemize}
    \item \textbf{Between Banks and Markets:} Through activities like asset securitization, asset-backed commercial paper (ABCP) conduits, and repurchase agreement (repo) operations.
    \item \textbf{Between Different Financial Systems:} Global integration means distress can spread internationally, e.g., US money market funds holding European bank debt.
\end{itemize}
These markets are often prone to shifts in sentiment, involving both rational revisions based on news about fundamentals and potentially irrational panics unrelated to fundamentals.

\subsection{Beliefs, Sentiments, and Agency Problems}
\begin{itemize}
    \item \textbf{Beliefs and Sentiments:} Psychological biases can influence asset prices and investor behavior, leading to "irrational exuberance" and bubbles. "Animal spirits," or fluctuations in confidence and trust, can drive excessive risk-taking.
    \item \textbf{Agency Problems:} Incentives within financial institutions can exacerbate risk. Loan officers might prioritize volume over quality, especially if institutional memory of past losses fades. Increased competition can pressure banks to lower lending standards to maintain market share.
    \item \textbf{Bank Governance:} Shareholders' interests may diverge from those of depositors or regulators. Because shareholders benefit from upside gains but have limited liability on the downside (their payoff resembles a call option on the bank's assets), they may have incentives to encourage higher risk-taking by the bank.
    \item \textbf{Deposit Insurance:} While protecting depositors, deposit insurance can create moral hazard, reducing the incentive for depositors to monitor banks and potentially encouraging banks to take on more risk (e.g., shifting from liquid assets to higher-risk assets), knowing depositors are covered.
\end{itemize}

\subsection{Impact of Low Interest Rates}
Prolonged periods of low monetary policy interest rates can contribute to systemic risk by:
\begin{itemize}
    \item Inducing a "search for yield," making low-risk assets less attractive and pushing banks and investors towards riskier assets.
    \item Potentially leading to an excessive softening of lending standards, particularly in sectors like real estate.
    \item Contributing to the growth of shadow banking activities as institutions seek higher returns outside the traditional, more regulated banking sector.
\end{itemize}

\section{Contagion Mechanisms}

Contagion is the process by which financial distress spreads from one institution or market to others, amplifying initial shocks. Key channels include:

\begin{itemize}
    \item \textbf{Contagion through Expectations:} Changes in expectations can become self-fulfilling. Rumors or negative sentiment can trigger bank runs (by retail depositors) or funding freezes (by wholesale investors), leading to liquidity crises even for solvent institutions. The default of one institution can trigger fears about counterparties, leading to a chain reaction. Derivatives and complex financial contracts amplify this interconnectedness.
    \item \textbf{Contagion through Liquidity Shortages:} If one institution faces liquidity problems and starts selling assets (potentially at fire-sale prices) or hoarding liquidity, it can negatively impact market prices and funding availability for others. This can lead to a broader credit crunch as banks reduce lending to conserve liquidity or meet capital requirements under stress.
    \item \textbf{Contagion to Other Markets (Spillovers):} Distress often spills over from one market segment to others (e.g., banking sector problems affecting equity and bond markets). Feedback loops between markets can amplify shocks.
    \item \textbf{Information Contagion/Reassessment:} Negative news about one institution (e.g., disclosure of large losses) can lead investors to reassess the health of similar institutions, causing a loss of confidence and funding even without direct exposures.
    \item \textbf{Cross-Country Contagion:} In a globalized system, distress spreads internationally via trade links, direct financial linkages (cross-border lending/investment), and correlated investor sentiment.
\end{itemize}
These channels often interact and reinforce each other, potentially leading to a cascade of failures. The magnitude of contagion depends on the initial fragility of the system, the degree of interconnectedness, and prevailing behaviors like herding.

\section{Real Effects of Systemic Crises}

Systemic financial crises have severe and persistent negative consequences for the real economy.

\subsection{Mechanisms of Impact}
\begin{itemize}
    \item \textbf{Credit Crunch:} Financial institutions reduce lending due to losses, capital constraints, funding difficulties, or increased risk aversion. This particularly affects credit-dependent borrowers like Small and Medium-sized Enterprises (SMEs).
    \item \textbf{Investment Decline:} Reduced credit availability and heightened economic uncertainty deter businesses from investing in capital projects.
    \item \textbf{Consumption Decline:} Households cut spending due to job losses, uncertainty about future income, reduced credit access, and negative wealth effects (falling asset prices like homes and stocks).
    \item \textbf{Increased Unemployment:} Businesses reduce production and investment, leading to layoffs, which further dampens consumption and economic activity.
\end{itemize}

\subsection{Fiscal Costs}
Governments often intervene during crises to stabilize the financial system and support the economy through:
\begin{itemize}
    \item Bailouts or capital injections for financial institutions.
    \item Liquidity support operations.
    \item Stimulus measures for the broader economy.
\end{itemize}
These interventions result in significant fiscal costs, leading to increased government debt, potentially crowding out other public spending, and having long-term implications for fiscal sustainability.

\subsection{Long-Term Effects}
Systemic crises can lower an economy's potential output persistently through reduced investment in physical and human capital, destruction of organizational capital, and hysteresis effects in the labor market. Cross-country evidence confirms that systemic crises typically lead to significant and lasting declines in economic output, though the magnitude varies depending on crisis severity and policy responses.

\subsection{Policy Responses to Mitigate Real Effects}
Effective policy responses are crucial to limit the real damage. These include:
\begin{itemize}
    \item \textbf{Monetary Policy:} Lowering interest rates, providing ample liquidity.
    \item \textbf{Fiscal Policy:} Targeted government spending, automatic stabilizers.
    \item \textbf{Financial Sector Policies:} Capital injections, guarantees, resolution mechanisms.
\end{itemize}
Early intervention is often key to preventing escalation. Interventions must be carefully designed to avoid exacerbating moral hazard or creating other unintended consequences.

\section{Measuring Systemic Risk}

Measuring systemic risk is essential for effective macroprudential policy but poses significant challenges.

\subsection{Importance and Objectives}
\begin{itemize}
    \item Develop summary indexes of overall system risk.
    \item Identify the cross-sectional contribution of individual institutions to systemic risk.
    \item Provide early warnings of rising vulnerabilities.
    \item Inform the calibration of macroprudential tools.
\end{itemize}

\subsection{Data Requirements and Limitations}
Ideal measurement requires detailed, high-frequency data on:
\begin{itemize}
    \item Financial institutions' balance sheets and risk exposures.
    \item Interconnections (credit exposures, derivatives, repo, interbank lending).
\end{itemize}
However, a complete map of financial connections is often unavailable, especially globally or for non-listed entities and emerging markets. Lack of high-frequency data hinders tracking rapid shifts in risk.

\subsection{Measurement Approaches}
Given data limitations, policymakers use a range of approaches:
\begin{itemize}
    \item \textbf{Macroeconomic Indicators:} Monitoring aggregate variables like credit-to-GDP gaps, asset price growth (potential bubbles), leverage ratios.
    \item \textbf{Market-Based Measures:} Using information from market prices, such as volatility indices (e.g., VIX), credit default swap (CDS) spreads, equity price correlations, measures of systemic expected shortfall (e.g., SRISK, CoVaR).
    \item \textbf{Accounting-Based Measures:} Using balance sheet data, such as capital adequacy ratios (Tier 1, leverage ratio), asset quality measures (non-performing loans), liquidity ratios (LCR, NSFR).
    \item \textbf{Network Models:} Attempting to map direct and indirect exposures between institutions to simulate contagion effects from defaults or stress events. These require detailed data and face challenges given the rarity of systemic events.
\end{itemize}
Practical measurement often involves combining indicators from multiple approaches and applying expert judgment.

\section{Financial Regulation and Systemic Risk}

Financial regulation aims to ensure the stability and efficiency of the financial system.

\subsection{Rationale for Regulation}
Regulation is justified by market failures, including:
\begin{itemize}
    \item \textbf{Asymmetric Information:} Between financial institutions and their customers/counterparties.
    \item \textbf{Externalities:} The failure of one institution can impose costs on others and the wider economy (systemic risk).
    \item \textbf{Moral Hazard:} Deposit insurance or expectations of bailouts can incentivize excessive risk-taking.
\end{itemize}
Without regulation, banks' private choices regarding capital, leverage, and risk might be inadequate from a societal perspective, leading to inefficiencies and instability.

\subsection{Microprudential Regulation}
Focuses on the safety and soundness of individual financial institutions.
\begin{itemize}
    \item \textbf{Objectives:} Reduce the risk of individual bank insolvency, protect depositors and taxpayers, promote stability at the firm level.
    \item \textbf{Tools:} Minimum capital requirements (risk-weighted and leverage-based), liquidity requirements (LCR, NSFR), restrictions on activities, supervisory oversight and monitoring, prompt corrective action frameworks.
    \item \textbf{Basel Accords (esp. Basel II):} Provided core principles for effective banking supervision, emphasizing microprudential soundness.
    \item \textbf{Limitations:} Primarily focuses on idiosyncratic risk, may not adequately address system-wide risks, and can sometimes have unintended consequences (e.g., encouraging risk migration to less regulated sectors or procyclical effects where tightening standards in a downturn exacerbates it).
\end{itemize}

\subsection{Macroprudential Regulation}
Focuses on the stability of the financial system as a whole, complementing microprudential regulation.
\begin{itemize}
    \item \textbf{Goals:} Limit the build-up of systemic risk across the financial cycle (time dimension) and across institutions (cross-sectional dimension); increase the resilience of the system to shocks.
    \item \textbf{Basel III Framework:} Incorporated macroprudential elements alongside strengthened microprudential standards.
    \item \textbf{Key Tools:}
        \begin{itemize}
            \item \textbf{Countercyclical Capital Buffers (CCyB):} Require banks to hold more capital during credit booms, releasable during downturns to absorb losses and sustain lending.
            \item \textbf{Dynamic Loan Loss Provisioning:} Setting aside provisions based on expected losses over the cycle, smoothing lending.
            \item \textbf{Loan-to-Value (LTV) and Debt-to-Income (DTI) Ratio Caps:} Limit leverage in household borrowing (especially mortgages) to prevent asset bubbles and reduce default risk.
            \item \textbf{Limits on Credit Growth:} Direct restrictions on lending expansion to control overheating.
            \item \textbf{Systemic Risk Surcharges:} Higher capital requirements for SIFIs.
            \item \textbf{Liquidity Regulation (LCR/NSFR):} Also has macroprudential benefits by reducing funding risk system-wide.
        \end{itemize}
\end{itemize}
Macroprudential policy aims to address the externalities and system-wide feedback loops that microprudential regulation alone cannot tackle effectively.

\section{Conclusion}

Systemic risk is an inherent feature of modern financial systems, driven by endogenous risk-taking, complex interlinkages, and behavioral factors, and amplified through various contagion channels. It poses a significant threat to financial stability and can inflict severe and lasting damage on the real economy. While challenging to measure precisely, understanding its drivers and propagation mechanisms is critical. Both microprudential regulation (focused on individual firm soundness) and macroprudential regulation (focused on system-wide stability) are necessary, complementary components of a framework designed to mitigate systemic risk and foster a resilient financial system capable of supporting sustainable economic growth.
