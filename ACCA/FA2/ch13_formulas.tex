\newpage 

\section*{Share Issues, Dividends, and Finance Costs}


\subsection{13.1 Capital Structure of Limited Liability Companies} 

A limited liability company is a legal entity where shareholders, as owners, are only liable for their invested capital. They appoint directors to manage the company. The shareholders earn a return through dividends or appreciation in share value, while the business operates as a separate legal entity.

A business's **capital structure** is the mix of **equity** (shares) and **debt** (loans) used for long-term financing. The primary sources are **ordinary (equity) shares**, which give ownership and voting rights, and **preference shares**, which offer a fixed dividend but usually no voting rights. Additionally, a company can raise capital through **borrowings/loan notes**, which represent a debt to the company.

During a **winding-up** (liquidation), a company's assets are sold to pay off its debts in a strict hierarchy: secured creditors are paid first, followed by preferential creditors (e.g., employees), then unsecured creditors, and finally, any remaining funds are distributed to shareholders, with preference shareholders paid before ordinary shareholders.


\subsection{ 13.2 Sources of Capital }

**Ordinary (equity) shares** represent ownership in a company.

1. They have a **nominal value**, but their market price can differ.
2. Shareholders have the right to a share of the company’s profits, paid out as **dividends**, although payment is not mandatory.
3. They also have **voting rights** on key company decisions.
4. In a liquidation, ordinary shareholders are the last to receive any remaining assets and may lose their entire investment.
5. When a company issues shares, it records the transaction with a double entry:

- **Debit** Bank (for cash received).
- **Credit** Ordinary Share Capital (for the nominal value).
- **Credit** Share Premium (for any amount received above the nominal value).

**Preference shares** are a hybrid form of capital with features of both equity and debt. Unlike ordinary shares, preference shareholders have the **right to a fixed annual dividend** that must be paid before any dividends are given to ordinary shareholders. However, they typically **do not have voting rights**. In a company liquidation, preference shareholders have priority over ordinary shareholders for the return of their investment.

There are two types of preference shares:

- **Redeemable preference shares** are classified as a **liability** because the company is obligated to repurchase them at a future date. Their dividends are treated as a **finance expense**.
    
- **Irredeemable preference shares** are a permanent form of capital and are classified as **equity** because the company has no obligation to redeem them.
    
The accounting entry for issuing preference shares depends on their type. For redeemable shares, you **debit Bank** and **credit a liability account** called "Redeemable preference shares." For irredeemable shares, you **debit Bank** and **credit an equity account** called "Irredeemable preference shares."

Instead of issuing shares, a company can raise capital through borrowings or loan notes, which are classified as liabilities. These carry a fixed interest rate payable to the lender and a principal amount that will be repaid (redeemed) on a specific future date. Unlike shareholders, lenders are creditors who can take legal action if they don't receive their interest payments. Borrowings can also be secured on a company's assets, meaning the lender has a legal right to claim and sell those assets to recover their money if the company defaults. The accounting entry for a new borrowing is to debit the Bank account for the cash received and credit the Borrowings liability account.

\subsection{13.3 Share Capital}

A rights issue is when a company offers its existing shareholders the **right to buy new shares**, typically at a price below the current market value. This is a method companies use to raise new funds while giving existing owners priority and maintaining their proportionate control, provided all shareholders participate.

The accounting entry for a rights issue is:

- **Debit Bank** for the total cash received.
    
- **Credit Ordinary Share Capital** for the total nominal value of the shares issued.
    
- **Credit Share Premium** for the amount received above the nominal value.
    

Key advantages include compliance with pre-emption rights, a cheaper administration cost than a public share issue, and the preservation of shareholder control if all parties take up their rights. Disadvantages include the risk of the share price falling and the potential for a change in shareholder control if not all shareholders participate.

A **bonus issue**, also known as a capitalization issue, is when a company gives new shares to existing shareholders for **free** and in proportion to their current holdings. Since no cash is raised, the new shares are funded from the company's existing capital, such as the share premium or retained earnings accounts. The total capital remains unchanged.

---

The double-entry to record a bonus issue is to:

- **Debit** either the **Share Premium** or **Retained Earnings** account. This reduces the balance of one of these accounts.
- **Credit** the **Ordinary Share Capital** account for the nominal value of the new shares issued.

For example, a "1 for 5" bonus issue means a shareholder receives one new share for every five shares they already hold.

---


**Advantages:**
- Increases the number of shares, which can lower the per-share price and make the shares more attractive to investors.
- Gives the company the appearance of being well-capitalized by increasing the share capital balance.
- Provides a return to shareholders without using cash.
- Does not dilute the ownership percentage of existing shareholders.

**Disadvantages:**
- The process is costly to administer.
- If retained earnings are used, it reduces the amount of profit available for future cash dividends.
- The company does not receive any cash from the issue.

There are several key terms related to a company's share capital, including its legal value, the total number of shares it's permitted to issue, and the portion of shares it has actually sold to investors.

---

\subsection{Share Capital Terminology}

- **Par Value (or Nominal Value):** This is the legal face value assigned to each share when a company is set up. The market price of a share can be very different from its par value. For example, a company may have shares with a par value of $1, but they could trade on the market for $50 each.
    
- **Authorised Share Capital:** This is the **maximum number of shares** that a company is legally permitted to issue, as specified in its constitutional documents. A company can change this limit with shareholder approval.
    
- **Issued Share Capital:** This is the number of shares that have been **sold to and are held by shareholders**. This number cannot exceed the authorized share capital.
    
- **Called-Up Share Capital:** This refers to the portion of the par value that a company has formally **requested from its shareholders**. For example, a company might issue shares with a \$1 par value but only "call up" 60\% of that value, asking shareholders to pay \$0.60 per share.
    
- **Paid-Up Share Capital:** This is the portion of the called-up share capital that shareholders have **actually paid**. Any unpaid amounts are known as "calls in arrears" and are a receivable for the company. The company’s statement of financial position will show the called-up share capital in equity and any calls in arrears as a current asset.


Dividends are an appropriation of a company's distributable profits to its shareholders. They are proposed by management and approved by shareholders, typically at an annual general meeting. For accounting purposes, dividends are only recorded in the financial statements when they are **paid** or **declared**. Proposed dividends are only disclosed in the notes to the accounts. The journal entry to record a paid dividend is to **debit Retained Earnings** and **credit Bank**.

---

 Dividends on Ordinary Shares

Ordinary shareholders are not guaranteed a dividend each year. The payment can be calculated as a percentage of profit for the year, a percentage of the par value of shares issued, or a fixed amount per share. Because these dividends are a distribution of profits, the entry reduces both the company's retained earnings and cash balance.

---

 Dividends on Irredeemable Preference Shares

These shares are classified as equity, and their dividends are also treated as a distribution of profit. The accounting entry for payment is the same as for ordinary shares: **debit Retained Earnings** and **credit Bank**. Preference dividends must be paid before any dividends are distributed to ordinary shareholders.

---

 Activity 5: Calculation and Reporting of Dividend

- **Calculate the total dividend payment:** The dividend is calculated as a percentage of the shares' par value.
    
    - Total dividend = 150,000 shares × \$1.50 par value × 5\% = **\$11,250**.
        
- **Financial statement reflection:** This dividend would **not** be reflected in the financial statements for the year ended 31 December 20X6. Since the dividend was declared on February 17, 20X7 (after the 20X6 year-end), it will only be recorded and accounted for in the financial statements for the year ended 31 December 20X7. It would, however, be disclosed as a proposed dividend in the notes to the 20X6 financial statements.



A company's capital and reserves, found in the equity section of its statement of financial position, represent the owners' interest. Reserves are funds held for specific purposes, are not cash, and are distinct from liabilities. They can be either **distributable** (like retained earnings, which can be used for dividends) or **non-distributable** (with restrictions on their use).

 Types of Reserves

---

 Share Premium

The **share premium** is the amount by which a share's issue price exceeds its nominal (par) value. It's a **non-distributable reserve** and cannot be used for dividends. Its limited uses include issuing bonus shares or covering specific company formation expenses.

---

Revaluation Surplus

The **revaluation surplus** records unrealised gains from revaluing non-current assets. It is a **non-distributable reserve** until the asset is sold, at which point the surplus is **realised** and transferred to retained earnings.

---
 Retained Earnings

**Retained earnings** represent the accumulated profits of a business that have not been distributed as dividends. This is a **distributable reserve**, meaning it can be used to pay future dividends. Portions of retained earnings may be transferred to other reserves for specific uses. When a revalued asset is sold, the realized revaluation surplus is transferred to retained earnings.



Movements on non-current assets for the year ended 30 September 20X6

% ---------------- Share Capital Ledger ----------------
\section*{Share Capital Ledger}

\begin{tabular}{@{}p{0.45\linewidth} p{0.45\linewidth}@{}}
\textbf{DR} & \textbf{CR} \\ \midrule
Share buyback / redemption \hfill \$XXX & Balance b/d \hfill \$XXX \\
& Issue of ordinary shares (cash) \hfill \$XXX \\
& Issue of preference shares \hfill \$XXX \\
& Rights issue \hfill \$XXX \\
& Bonus issue (transfer from reserves) \hfill \$XXX \\[3pt]
\multicolumn{2}{c}{\hrulefill} \\
Total \hfill \$XXX & Total \hfill \$XXX \\[6pt]
& Balance b/d \hfill \$XXX \\
\end{tabular}

\vspace{1cm}

% ---------------- Reserves / Equity Ledger ----------------
\section*{Reserves / Equity Ledger (Retained Earnings, Share Premium)}

\begin{tabular}{@{}p{0.45\linewidth} p{0.45\linewidth}@{}}
\textbf{DR} & \textbf{CR} \\ \midrule
Dividends declared \hfill \$XXX & Balance b/d \hfill \$XXX \\
Transfer to bonus issue \hfill \$XXX & Profit for the year (from P\&L) \hfill \$XXX \\
& Share premium on issue of shares \hfill \$XXX \\
& Revaluation surplus (OCI) \hfill \$XXX \\[3pt]
\multicolumn{2}{c}{\hrulefill} \\
Total \hfill \$XXX & Total \hfill \$XXX \\[6pt]
& Balance b/d \hfill \$XXX \\
\end{tabular}

\vspace{1cm}

% ---------------- Finance Costs Ledger ----------------
\section*{Finance Costs Ledger (Loan Notes / Borrowings)}

\begin{tabular}{@{}p{0.45\linewidth} p{0.45\linewidth}@{}}
\textbf{DR} & \textbf{CR} \\ \midrule
Cash/Bank (interest paid) \hfill \$XXX & Balance b/d (opening accrual) \hfill \$XXX \\
Balance c/d (closing accrual) \hfill \$XXX & Interest expense accrued \hfill \$XXX \\
Loss on early redemption of loan notes \hfill \$XXX & Loan finance cost charged \hfill \$XXX \\
& Premium on redemption \hfill \$XXX \\[3pt]
\multicolumn{2}{c}{\hrulefill} \\
Total \hfill \$XXX & Total \hfill \$XXX \\[6pt]
Balance b/d \hfill \$XXX & \\
\end{tabular}

\subsection*{1. Share Issues}

\subsubsection*{(a) Rights Issue}

\textbf{Cash Raised Formula:}
\[
\text{Cash Raised} = \text{Number of Existing Shares} \times \frac{\text{New Shares}}{\text{Existing Shares}} \times \text{Issue Price per Share}
\]

\textbf{Accounting Entries:}
\[
\begin{aligned}
\text{DR Bank} &: \ \text{Number of New Shares} \times \text{Issue Price} \\
\text{CR Ordinary Share Capital} &: \ \text{Number of New Shares} \times \text{Par Value} \\
\text{CR Share Premium} &: \ \text{Number of New Shares} \times (\text{Issue Price} - \text{Par Value})
\end{aligned}
\]

\subsubsection*{(b) Bonus Issue}

\textbf{Number of New Shares Formula:}
\[
\text{Number of New Shares} = \text{Existing Shares} \times \text{Bonus Ratio}
\]

\textbf{Increase in Share Capital Formula:}
\[
\text{Increase in Ordinary Share Capital} = \text{Number of New Shares} \times \text{Par Value}
\]

\textbf{Accounting Entries:}
\[
\begin{aligned}
\text{DR Share Premium or Retained Earnings} \\
\text{CR Ordinary Share Capital}
\end{aligned}
\]

\subsection*{2. Dividends and Finance Costs}

\subsubsection*{(a) Ordinary Share Dividends}

\textbf{Percentage of Profit:}
\[
\text{Dividend Amount} = \text{Net Profit} \times \text{Dividend Percentage}
\]

\textbf{Percentage of Par Value:}
\[
\text{Dividend Amount} = \text{Number of Shares} \times \text{Par Value} \times \text{Dividend Percentage}
\]

\textbf{Per Share:}
\[
\text{Dividend Amount} = \text{Number of Shares} \times \text{Dividend per Share}
\]

\textbf{Accounting Entries (when paid):}
\[
\begin{aligned}
\text{DR Retained Earnings} \\
\text{CR Bank}
\end{aligned}
\]

\subsubsection*{(b) Finance Costs}

\textbf{Redeemable Preference Shares:}
\[
\text{Annual Finance Cost} = \text{Number of Shares} \times \text{Par Value} \times \text{Dividend Percentage}
\]

\textbf{Interest on Borrowings:}
\[
\text{Annual Finance Cost} = \text{Principal Amount} \times \text{Interest Rate}
\]

\textbf{Accounting Entries (when paid):}
\[
\begin{aligned}
\text{DR Finance Costs} \\
\text{CR Bank}
\end{aligned}
\]
