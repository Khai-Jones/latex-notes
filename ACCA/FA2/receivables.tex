\myBoxedTopic{ACCOUTNING FOR BUSINESS TRANSACTIONS}{July 30th, 2025}

\scriptsize


    \textbf{The Financial Accounting syllabus introduces five business activities:}
\mytextbox{
    \begin{itemize}[itemsep=-0.2em, rightmargin=2.6cm]
        \item \highlight[2.5pt]{Selling Goods or Services:} This can involve immediate payment from customers (Cash Sales) or a promise of payment at a later date (Credit Sales). 
        \item \highlight[2.5pt]{Customer Returning Goods:} When customers bring back items they bought (Sales Returns).   
        \item \highlight[2.5pt]{Buying Goods or Services:}  This can involve paying immediately with cash (Cash Purchases) or promising to pay later (Credit Purchases). 
        \item \highlight[2.5pt]{Returning Goods to Suppliers:} When our business sends back items we bought (Purchase Returns). 
        \item \highlight[2.5pt]{Small Cash Payments:} Handling minor, everyday expenses using a special small cash fund (Petty Cash transactions). 
\end{itemize}
}

  The business activities can be also be furter split into three groups: activites that affect cash, activites that affect credit and activites that affect petty cash. Cash sales and purchases, credit sales and purchases, petty expenses. 

  \textbf{Each of these business transactions goes through a consistent three-step process.}

  \hlOrange{\textbf{Firstly}}, when a business transaction takes place, it creates a source document. This document acts as an original piece of evidence for the transaction. 
     
  \hlOrange{\textbf{Secondly}}, this source document is used to write down the transaction in a journal entry. The journal entry is a chronological diary where every business transaction is initially recorded. In this entry we figure out what specific accounts (like `Cash', `Sales Revenue', or `Accounts Payable') are affected and whether they need to increase or decrease.  This is important because for each transaction, *at least* two accounts will be affected. One will be debited and at least one will be credited for the exact amount. 
     
  \hlOrange{\textbf{Lastly}}, after a journal entry is made, its details are posted to the affected inidividual legder accounts. Each individual ledger account resides within a document called the general ledger. A simple way to think of this is as a binder, where each page is an individual ledger account and the binder is the general ledger account. When an entry is *posted*, we simply take the debit and credit amounts from the journal entry and apply them to the correct sides of their respective individual accounts in the general ledger. This updates the balance of each account, so we always know how much money we have, how much customers owe us, what our expenses arc, and so on.

  



\newpage
\subsection{Receivables}

Receivables are debts owed to a business by its customers, typically originating from previous credit transactions. 



\section{Formulas}

\subsection{Allowance for Doubtful Debts}

\[
\text{Allowance} = \sum_{i=1}^{m} (S_i \times \text{Allowance Rate}_i) + \sum_{j=1}^{n} (G_j \times \text{Allowance Rate}_j)
\]

\subsubsection{Step 1: Writing off irrecoverable (bad) debt}

\[
\text{DR Irrecoverable Debts Expense (P/L)} = \text{amount}
\]
\[
\text{CR Trade Receivables (SFP)} = \text{amount}
\]

\subsubsection{Step 2: Calculating the allowance}

1. Identify specific receivables requiring 100\% allowance:
\[
\text{Allowance} = \text{Receivable Balance} \times 100\%
\]

2. Apply a percentage allowance to the remaining receivables:
\[
\text{Allowance} = \text{Remaining Receivables} \times \text{Allowance Rate}
\]

3. Closing allowance = total of specific + general allowances.

\subsubsection{Step 3: Adjustment for allowance at year-end}

If Closing Allowance $>$ Opening Allowance:
\[
\text{DR Irrecoverable Debts Expense}, \quad \text{CR Allowance for Doubtful Debts}
\]

If Closing Allowance $<$ Opening Allowance:
\[
\text{DR Allowance for Doubtful Debts}, \quad \text{CR Irrecoverable Debts Expense}
\]

\subsubsection{Step 4: When a debt is written off after an allowance was made}

\[
\text{DR Irrecoverable Debts Expense}, \quad \text{CR Trade Receivables}
\]

Simultaneously, reduce the allowance account (credit to expense), so the debt is not double-counted.

\subsubsection{Formula Summary}

\[
\text{Closing Allowance} = \text{Specific Doubtful Debts} + (\text{Remaining Receivables} \times \text{General \%})
\]

\[
\text{Adjustment} = \text{Closing Allowance} - \text{Opening Allowance}
\]

---

\section{Accounting Equation}

\[
\text{Capital} = \text{Assets} - \text{Liabilities}
\]

\[
\text{Capital at Year-end} = \text{Opening Capital} + \text{Capital Introduced} + \text{Profit} - \text{Drawings}
\]

Rearranging for Profit:

\[
\text{Profit} = (\text{Assets} - \text{Liabilities}) - (\text{Opening Capital} + \text{Capital Introduced}) + \text{Drawings}
\]

---

\section{Control Account Pro formas}

\subsection{Trade Receivables}

\[
\text{Credit Sales} = \text{Cash Received from Customers} + \text{Closing Receivables} - \text{Opening Receivables}
\]

\[
\text{Closing Receivables} = \text{Credit Sales} - \text{Cash Received} + \text{Opening Receivables}
\]

\subsection{Trade Payables}

\[
\text{Credit Purchases} = \text{Cash Paid to Suppliers} + \text{Closing Payables} - \text{Opening Payables}
\]

\[
\text{Closing Payables} = \text{Credit Purchases} - \text{Cash Paid to Suppliers} + \text{Opening Payables}
\]

\subsection{Bank (Cash) Account}

\[
\text{Opening Bank Balance} + \text{Cash Receipts} = \text{Cash Payments} + \text{Closing Bank Balance}
\]

\[
\text{Drawings} = \text{Opening Bank Balance} + \text{Cash from Customers} - \text{Cash to Suppliers} - \text{Other Expenses Paid} - \text{Closing Bank Balance}
\]

---

\section{Mark-up, Margin and Inventory Formulas}

\subsection{1. Mark-up (profit on cost)}

\[
\text{Sales} = \text{COGS} \times \frac{100 + \text{Mark-up \%}}{100}
\]

\[
\text{COGS} = \text{Sales} \times \frac{100}{100 + \text{Mark-up \%}}
\]

\[
\text{Gross Profit} = \text{Sales} \times \frac{\text{Mark-up \%}}{100 + \text{Mark-up \%}}
\]

\subsection{2. Margin (profit on sales)}

\[
\text{Sales} = \text{COGS} \times \frac{100}{100 - \text{Margin \%}}
\]

\[
\text{COGS} = \text{Sales} \times \frac{100 - \text{Margin \%}}{100}
\]

\[
\text{Gross Profit} = \text{Sales} \times \frac{\text{Margin \%}}{100}
\]

\subsection{3. Inventory}

\[
\text{COGS} = \text{Opening Inventory} + \text{Purchases} - \text{Closing Inventory}
\]

\[
\text{Closing Inventory} = \text{Opening Inventory} + \text{Purchases} - \text{COGS}
\]

\[
\text{Inventory Lost or Drawn for Personal Use} = \text{Expected Closing Inventory} - \text{Actual Closing Inventory}
\]

---

\subsection{Decision Aid}

\begin{itemize}
    \item If percentage is based on cost $\rightarrow$ use Mark-up formulas.
    \item If percentage is based on sales $\rightarrow$ use Margin formulas.
    \item If the question involves missing stock or movement $\rightarrow$ use Inventory formulas.
\end{itemize}




