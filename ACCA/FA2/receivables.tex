\myBoxedTopic{ACCOUTNING FOR BUSINESS TRANSACTIONS}{July 30th, 2025}

\scriptsize


    \textbf{The Financial Accounting syllabus introduces five business activities:}
\mytextbox{
    \begin{itemize}[itemsep=-0.2em, rightmargin=2.6cm]
        \item \highlight[2.5pt]{Selling Goods or Services:} This can involve immediate payment from customers (Cash Sales) or a promise of payment at a later date (Credit Sales). 
        \item \highlight[2.5pt]{Customer Returning Goods:} When customers bring back items they bought (Sales Returns).   
        \item \highlight[2.5pt]{Buying Goods or Services:}  This can involve paying immediately with cash (Cash Purchases) or promising to pay later (Credit Purchases). 
        \item \highlight[2.5pt]{Returning Goods to Suppliers:} When our business sends back items we bought (Purchase Returns). 
        \item \highlight[2.5pt]{Small Cash Payments:} Handling minor, everyday expenses using a special small cash fund (Petty Cash transactions). 
\end{itemize}
}

  The business activities can be also be furter split into three groups: activites that affect cash, activites that affect credit and activites that affect petty cash. Cash sales and purchases, credit sales and purchases, petty expenses. 

  \textbf{Each of these business transactions goes through a consistent three-step process.}

  \hlOrange{\textbf{Firstly}}, when a business transaction takes place, it creates a source document. This document acts as an original piece of evidence for the transaction. 
     
  \hlOrange{\textbf{Secondly}}, this source document is used to write down the transaction in a journal entry. The journal entry is a chronological diary where every business transaction is initially recorded. In this entry we figure out what specific accounts (like `Cash', `Sales Revenue', or `Accounts Payable') are affected and whether they need to increase or decrease.  This is important because for each transaction, *at least* two accounts will be affected. One will be debited and at least one will be credited for the exact amount. 
     
  \hlOrange{\textbf{Lastly}}, after a journal entry is made, its details are posted to the affected inidividual legder accounts. Each individual ledger account resides within a document called the general ledger. A simple way to think of this is as a binder, where each page is an individual ledger account and the binder is the general ledger account. When an entry is *posted*, we simply take the debit and credit amounts from the journal entry and apply them to the correct sides of their respective individual accounts in the general ledger. This updates the balance of each account, so we always know how much money we have, how much customers owe us, what our expenses arc, and so on.

  



\newpage
\subsection{Receivables}

Receivables are debts owed to a business by its customers, typically originating from previous credit transactions. 


\subsection{Formulas}

$\text{Allowance} = \sum_{i=1}^{m} (S_i \times \text{Allowance Rate}_i) + \sum_{j=1}^{n} (G_j \times \text{Allowance Rate}_j)$
