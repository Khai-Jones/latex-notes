\section*{Accruals and Prepayments}

\subsection*{1. Accruals}

\subsubsection*{Accrual Amount Formula}

This formula estimates the value of an expense incurred but not yet paid or invoiced by the end of an accounting period:

\[
\text{Accrual Amount} = \left( \frac{\text{Total Cost of Last Invoice}}{\text{Number of Months in Last Invoice}} \right) \times \text{Number of Months Accrued}
\]

\textbf{Example:}  
If electricity costs \$7,230 for 3 months, and 2 months need to be accrued:  
\[
\frac{7,230}{3} \times 2 = 4,820
\]

\subsubsection*{Journal Entry Formulas for Accruals}

\paragraph{Creating an Accrual (at year-end)}  
\[
\begin{aligned}
\text{DR Individual Expense Account} \\
\text{CR Accruals Account}
\end{aligned}
\]

\paragraph{Reversing an Accrual (in the next period)}  
\[
\begin{aligned}
\text{DR Accruals Account} \\
\text{CR Individual Expense Account}
\end{aligned}
\]

\paragraph{Recording the Expense Payment}  
\[
\begin{aligned}
\text{DR Individual Expense Account} \\
\text{CR Bank or Payables Account}
\end{aligned}
\]

\subsection*{2. Prepayments}

A prepayment is an asset recognized when a business pays for an expense in the current period that relates to a future period. The purpose is to match expenses to the correct accounting period.

\subsubsection*{Prepayment Amount Formula}

\[
\text{Prepayment Amount} = \text{Total Amount Paid} \times \frac{\text{Number of Months Prepaid}}{\text{Total Months Covered by Payment}}
\]

\textbf{Example 1:} A rent payment of \$1,200 covers 3 months (Dec, Jan, Feb). At Dec 31 year-end, 2 months (Jan, Feb) are prepaid:
\[
1,200 \times \frac{2}{3} = 800
\]

\textbf{Example 2:} A payment of \$4,875 covers 12 months (Sept X2–Aug X3). At Dec 31 year-end, 8 months (Jan–Aug X3) are prepaid:
\[
4,875 \times \frac{8}{12} = 3,250
\]

\subsubsection*{Expense for the Period Formula}

\[
\text{Expense for the Period} = \text{Total Amount Paid} - \text{Prepayment Amount}
\]

Alternative form:
\[
\text{Expense for the Period} = \text{Total Amount Paid} \times \frac{\text{Months in Current Period}}{\text{Total Months Covered}}
\]

\textbf{Example:} For the \$1,200 rent payment (Dec 1 covering Dec–Feb):  
\[
1,200 \times \frac{1}{3} = 400
\]  
The total expense for 20X2 could include multiple payments (e.g. June + September + December portions).

\subsubsection*{Journal Entry Formulas for Prepayments}

\paragraph{Recording the Expense Payment}
\[
\begin{aligned}
\text{DR Individual Expense Account} \\
\text{CR Bank / Trade Payables}
\end{aligned}
\]

\paragraph{Creating a Prepayment (at year-end)}
\[
\begin{aligned}
\text{DR Prepayment Account (Asset)} \\
\text{CR Individual Expense Account}
\end{aligned}
\]

\paragraph{Reversing a Prepayment (in the next period)}
\[
\begin{aligned}
\text{DR Individual Expense Account} \\
\text{CR Prepayment Account (Asset)}
\end{aligned}
\]
