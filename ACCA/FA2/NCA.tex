\section{Non-current Assets}

Depreciation is the wear and loss of an asset’s value over time due to its usage and consumption. Its purpose is to match the asset’s revenue to its expenses. 
The depreciation charge for the year is debited to the depreciation account in the statement of profit or loss and the corresponding credit is to the accumulated depreciation account which reduces the asset’s carrying value in the statement of financial position.
The straight-line method and the reducing balance method. The straight-line method charges in equal amount of depreciation each year, while the diminishing-balance method charges a higher amount in the assets early years and a lower amount later on. 
The business sets it’s own depreciation policy, which includes it’s chosen depreciation method, and a pro-rata policy for calculating depreciation proportionally in the years of acquisition and disposal. IAS16 requires that businesses regularly review their depreciation methods, useful lives, and residual values regularly as these are all estimates that change over-time. 
