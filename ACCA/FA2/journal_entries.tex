

\subsection*{Journal Entry: Cash Sales}

\vspace{1em}

\begin{center} 
\begin{tabular}{@{} l l l l r @{}}
\toprule
& \textbf{Account Title} & \textbf{Category} & \textbf{Explanation} & \textbf{Amount (\$)} \\
\midrule
Dr & Cash & Asset & Cash increased & \hfill 1,000 \\
 & \quad  & \\
Cr & Sales Revenue & Income & Sales increased& \hfill (1,000) \\
\bottomrule
\end{tabular}
\end{center}
\vspace{1em}
\textit{Journal Entry: Cash Sales}

\vspace{1em}

\begin{center} 
\begin{tabular}{@{} l l l l r @{}}
\toprule
& \textbf{Account Title} & \textbf{Category} & \textbf{Explanation} & \textbf{Amount (\$)} \\
\midrule
Dr & Trade Receivables & Asset & Trade Receivables increased & 1,000 \\
 & \quad  & \\
Cr & Sales Revenue & Income & Sales increased& (1,000) \\
\bottomrule
\end{tabular}
\end{center}
\vspace{1em}
\textit{Journal Entry: Trade Receivables}

\begin{center} 
\begin{tabular}{@{} l l l l r @{}}
\toprule
& \textbf{Account Title} & \textbf{Category} & \textbf{Explanation} & \textbf{Amount (\$)} \\
\midrule
Dr & Sales as Sales returns & Income & Sales decreased & 1,000 \\
 & \quad  & \\
Cr & Trade Receivables & Asset & Receivables decreased & (1,000) \\
\bottomrule
\end{tabular}
\end{center}
\vspace{1em}
\textit{Journal Entry: Sales Return}

\begin{center} 
\begin{tabular}{@{} l l l l r @{}}
\toprule
& \textbf{Account Title} & \textbf{Category} & \textbf{Explanation} & \textbf{Amount (\$)} \\
\midrule
Dr & Bank/Cash & Asset & Bank/Cash increased & 1,000 \\
 & \quad  & \\
Cr & Trade Receivables & Asset & Receivables decreased & (1,000) \\
\bottomrule
\end{tabular}
\end{center}
\vspace{1em}
\textit{Journal Entry: Receipts from customers}


\begin{center} 
\begin{tabular}{@{} l l l l r @{}}
\toprule
& \textbf{Account Title} & \textbf{Category} & \textbf{Explanation} & \textbf{Amount (\$)} \\
\midrule
Dr & Purchases & Expense & Purchases increased & 1,000 \\
 & \quad  & \\
Cr & Cash/Bank & Asset & Cash decreased & (1,000) \\
\bottomrule
\end{tabular}
\end{center}
\vspace{1em}
\textit{Journal Entry: Cash purchases}


\begin{center} 
\begin{tabular}{@{} l l l l r @{}}
\toprule
& \textbf{Account Title} & \textbf{Category} & \textbf{Explanation} & \textbf{Amount (\$)} \\
\midrule
Dr & Purchases & Expense & Purchases increased & 1,000 \\
 & \quad  & \\
Cr & Trade Payables & Liability & Payables increased & (1,000) \\
\bottomrule
\end{tabular}
\end{center}
\vspace{1em}
\textit{Journal Entry: Credit purchases}


\begin{center} 
\begin{tabular}{@{} l l l l r @{}}
\toprule
& \textbf{Account Title} & \textbf{Category} & \textbf{Explanation} & \textbf{Amount (\$)} \\
\midrule
Dr & Trade Payables & Liability & Payables decreased & 1,000 \\
 & \quad  & \\
Cr & Purchases & Expense & Purchase decreased & (1,000) \\
\bottomrule
\end{tabular}
\end{center}
\vspace{1em}
\textit{Journal Entry: Purchase Returns}

\begin{center} 
\begin{tabular}{@{} l l l l r @{}}
\toprule
& \textbf{Account Title} & \textbf{Category} & \textbf{Explanation} & \textbf{Amount (\$)} \\
\midrule
Dr & Trade Payables & Liability & Payables decreased & 1,000 \\
 & \quad  & \\
Cr & Bank/Cash & Asset & Cash decreased & (1,000) \\
\bottomrule
\end{tabular}
\end{center}
\vspace{1em}
\textit{Journal Entry: Payment to Suppliers}


\subsection{Irrecoverable Debts}

\begin{center} 
\begin{tabular}{@{} l l l l r @{}}
\toprule
& \textbf{Account Title} & \textbf{Category} & \textbf{Explanation} & \textbf{Amount (\$)} \\
\midrule
Dr & Irrecoverable debt expenses account & expenses & Irrecoverable debt expenses increased & 1,000 \\
 & \quad  & \\
Cr & Trade Receivables & Asset & Receivables decreased & (1,000) \\
\bottomrule
\end{tabular}
\end{center}
\vspace{1em}
\textit{Journal Entry: Payment to Suppliers}

\subsubsection{Subsequent Recovery of Irrecoverable Debt}

Step 1: Reverse the irrecoverable debt write-off 

\begin{center} 
\begin{tabular}{@{} l l l l r @{}}
\toprule
& \textbf{Account Title} & \textbf{Category} & \textbf{Explanation} & \textbf{Amount (\$)} \\
\midrule
Dr & Receivables & Asset & Receivables increased & 1,000 \\
 & \quad  & \\
Cr & Irrecoverable debts expense account & Expense & Irrecoverable debts expensed decreased & (1,000) \\
\bottomrule
\end{tabular}
\end{center}
\vspace{1em}
\textit{Journal Entry: Payment to Suppliers}

Since the balance owed has been paid, the amount is not irrecoverable. Therefore, an adjustment to reverse te earlier write-off is made. 

Step 2: Record the Receipts

\begin{center} 
\begin{tabular}{@{} l l l l r @{}}
\toprule
& \textbf{Account Title} & \textbf{Category} & \textbf{Explanation} & \textbf{Amount (\$)} \\
\midrule
Dr & Bank & Asset & Bank increased & 1,000 \\
 & \quad  & \\
Cr & Receivables & Asset & Receivables decreased & (1,000) \\
\bottomrule
\end{tabular}
\end{center}
\vspace{1em}
\textit{Journal Entry: Payment to Suppliers}

Therefore, the net effect of the above two entries is Dr\. Bank, Cr. Irrecoverable debits expense account. 


\subsection{Allowance for irrecoverable debts}

\begin{enumerate} 
    \item Calculate the closing allowance for the allowance for irrecoverable debts at the year-end. 
    \item Calculate the difference between the closing allowance and the opening allowance (brought forward from the previous accounting period) 
    \item The difference is posted as a journal entry to the allowance for irrecoverable debits ledger. The corresponding account is the irrecoverable debts expense account. 
\end{enumerate}

If the closing allowance is more than the opening allowance, the double entry to record the adjustment is: 

\begin{center} 
\begin{tabular}{@{} l l l l r @{}}
\toprule
& \textbf{Account Title} & \textbf{Category} & \textbf{Explanation} & \textbf{Amount (\$)} \\
\midrule
Dr & Irrecoverable debts expense & Expense & Bank debt increased & 1,000 \\
 & \quad  & \\
Cr & Allowance for irrecoverable debts & Asset & Receivables decreased & (1,000) \\
\bottomrule
\end{tabular}
\end{center}
\vspace{1em}
\textit{Journal Entry: Payment to Suppliers}

Since it has been identified that the closing allowance is more than the opening allowance, the difference is posted as an irrecoverable debts expense in the statement of profit or loss (in the same way as an irrecoverable debt written off). 

If the closing allowance calculated is less than the opening allowance, the double entry to record the adjustment is: 


\begin{center} 
\begin{tabular}{@{} l l l l r @{}}
\toprule
& \textbf{Account Title} & \textbf{Category} & \textbf{Explanation} & \textbf{Amount (\$)} \\
\midrule
Dr & Allowance for irrecoverable debts & Asset & Receivables increased & 1,000 \\
 & \quad  & \\
Cr & Irrecoverable debt expense & Expense & Irrecoverable debts expense decreased & (1,000) \\
\bottomrule
\end{tabular}
\end{center}
\vspace{1em}
\textit{Journal Entry: Payment to Suppliers}

Since the closing allowance is less than the opening allowance, the difference is posted to decrease the irrecoverable debt expense. The reduced expense will be shown in the statement of profit or loss.

(Note – while the Allowance for irrecoverable debts is described as an asset account, it is a negative asset, as it reduces the value of trade receivables in the statement of financial position.)

\subsection{Inventory} 

The record of inventory and cost of goods sold are made at the end of the year using journals. The objective of the double entries is to:

    Ensure the Inventory account reflects the closing inventory valuation
    Cost of goods sold account is created and reflects the correct amount

To achieve these objectives, there are three double-entry steps to make:

1. Remove the Opening Inventory

Opening inventories are removed and transferred to the Cost of goods sold account. This entry is necessary because the opening inventories are now used to generate sales in the current accounting period.

\begin{center} 
\begin{tabular}{@{} l l l l r @{}}
\toprule
& \textbf{Account Title} & \textbf{Category} & \textbf{Explanation} & \textbf{Amount (\$)} \\
\midrule
Dr & Cost of goods sold & Expense & Opening inventory cost now included as expenses & 1,000 \\
 & \quad  & \\
Cr & Inventory & Asset & Inventory decreased & (1,000) \\
\bottomrule
\end{tabular}
\end{center}
\vspace{1em}
\textit{Journal Entry: Payment to Suppliers}

The cost of opening inventories is reflected as a current-year expense in the Statement of Profit or Loss.

2. Close off the Purchases Account

A business makes purchases for inventory for resale. The cost is debited to the Purchases account and credited to cash/payables at the point of purchase. At year-end, the amount in the Purchases account is closed off and transferred to the Cost of Goods Sold.


\begin{center} 
\begin{tabular}{@{} l l l l r @{}}
\toprule
& \textbf{Account Title} & \textbf{Category} & \textbf{Explanation} & \textbf{Amount (\$)} \\
\midrule
Dr & Cost of goods sold & Expense & Purchases is transferred to COGS & 1,000 \\
 & \quad  & \\
Cr & Purchases & Expense & Purchases is closed off & (1,000) \\
\bottomrule
\end{tabular}
\end{center}
\vspace{1em}
\textit{Journal Entry: Payment to Suppliers}


3. Post the Closing Inventory

The balance in the inventory account at year-end should reflect the value of closing inventory. The closing balance is presented in the statement of financial position as a current asset.

Since closing inventories are items purchased that are not sold in the accounting period, their cost should not be reflected as an expense in the Cost of goods sold account (SPL). Therefore, the value of closing inventory is transferred out of expenses and reflected as Closing inventory in the Statement of Financial Position.

\begin{center} 
\begin{tabular}{@{} l l l l r @{}}
\toprule
& \textbf{Account Title} & \textbf{Category} & \textbf{Explanation} & \textbf{Amount (\$)} \\
\midrule
Dr & Inventory & Asset & Inventory is increased & 1,000 \\
 & \quad  & \\
Cr & Cost of goods sold & Expense & Costs decreased & (1,000) \\
\bottomrule
\end{tabular}
\end{center}
\vspace{1em}
\textit{Journal Entry: Payment to Suppliers}

The value of closing inventory will be next year's opening inventory value.




