\chapter{Recording Transactions and Events}

\section{IAS 37 Provisions, Contingent Liabilities and Contingent Assets}

\subsection{Provisions}

(IAS37) A provision is a liability of uncertain timing or amount and should be recognised when all the following criteria is satisified:

\subsubsection{Recognition Criteria}

Provisions are recognised upon the satisifaction of three criteria: 

\begin{enumerate}
    \item A present obligation exists due to a past event.
    \item An outflow of economic resources to settle the obligation is probable, meaning that there is a greater than 50\% chance of it occuring.
    \item A reliable estimate of the obligation can be made.
\end{enumerate}

If any of the above conditions are not fulfilled, a provision must not be recognised.
The main difference between a provision and a liability is the extent of uncertainty involved. IAS37 states that the amount of a provision should be equal to "the best estimate of the 
expenditure required to settle the present obligation at the end of the reporting period". 


The amount of the provision must be the best estimate of the expenditure required to settle the obligation concerned. This is of course a matter of judgement and may require advice from independent experts. Note that: 

\begin{enumerate}[a.]
    \item The "best estimate" of the required expenditure is the amount that the entity would rationally pay to settle the obligation or transfer it to a third party. 
    \item If the effect of the time value of money is material, the amount of a provision should be calculated as the present value of the expenditure required to settle the obligation. 
    \item Future events that may affect the amount required to settle an obligation should be taken into account when measuring a provision, so long as there is sufficient objective evidence that these events will occur. For instance, the estimated costs of cleaning up environmental damage might be reduced by anticipated changes in technology. 
\end{enumerate}

If a single obligation is being measured and there are several possible outcomes, the best estimate of the required expenditure is usually the cost of the most likely outcome.

But if most of the other possible outcomes would involve a higher cost, the best estimate of the expenditure will be a higher amount. Similarly, if most of the other possible outcomes would involve a lower cost, the best estimate of the
expenditure will be a lower amount.

In cases where a provision relates to a large population of items, the amount of the provision should be estimated by calculating the "expected value" of the obligation at the
end of the reporting period. This statistical method of estimation involves weighting all possible outcomes by their associated probabilities.

Example 1: 

In 20X6, Benedict was sued for damages by a significant customer for breach of contract. 
In March 20X7, the court ruled in favour of the customer but deferred its ruling on the number 
of damages until June. For legal advice in defence of this claim, Benedict paid \$15,000 in 20X6;
a further \$20,000 has been incurred to date (to be paid once the matter is settled in June), and Benedict expects 
to pay an additional \$10,000 before the case is wholly settled.

Benedict’s legal adviser thinks that Benedict will be required to reimburse the customer's legal costs, which he estimates will be as much as 
Benedict’s. Based on the level of damages claimed, he also believes these are likely to be in the region of \$250,000 to \$300,000.

Analysis of best estimate:

Benedict should provide for the following: 

\begin{itemize}
    \item its legal costs incurred after the end of the reporting period (\$15,000 incurred in 20X6 is already expensed, so not considered in calculating the provision.)
    \item the best estimate of the customer's legal costs 
    \item the best estimate of the damages. This is the most subjective. As the estimated range of the outcome is relativialy narrow, any amount in this rangce may be considered as good an estimate as any other. However, a midpoint may be selected as the lower end of the range may be considered imprudent, and the upper end over prudent. 
\end{itemize}


Benedict's legal costs: 20,000 + 10,000 = \$30,000

Customer's legal costs: 15,000 + 20,000 + 10,000 = \$45,000

For the award of damages (mid-point of range) = \$275,000 - \$350,000

\section{Contingent Liabilities}



