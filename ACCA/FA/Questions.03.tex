\section{Introduction to the Regulatory Framework}


The primary roles of the International Accounting Standards Board and the International Sustainability Standards Board within the IFRS Foundation's governance structure are to develop and publish accounting standards and develop a baseline of sustainability-related disclouse standards.

Pubilc accountability refers to being subject to punishment or consequences in the case of failure or inadequate performance in carrying out tasks. In the case of IFRS standard-setting, they are held accountable by a monitoring board of public authorities (IFRS Foundation Monitoring Board). 


It is said that the "regulatory bodies do not have the power to force companies to comply" because, legally, they do not. IFRS is much more of a recommendation and implies that global adoption of their guidance can be ignored if chosen to do so.  


National accounting bodies can adopt IFRSs as standard or use it as a basis for developing guidance for their own country. They can also develop their own requirements but compare them to IFRSs to determine if their standards are sufficient. 
