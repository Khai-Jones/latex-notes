\section{5.1 Recording Business Transaction}

The recording business transactions module comprises of four subtopics. 

\begin{enumerate}
    \item Introduction to Recording Business Transactions 
    \item  Sales Tax
    \item Discount Received and Allowed
    \item  Bank Reconciliations
\end{enumerate}

\subsection{Introduction to Recording Business Transactions}

\subsubsection{Context: Types of Business Transactions}

Businesses engage in day-to-day operations to sustain the business' life cycle. The monetary value of these operations are referred to as business transactions. These transactions are important because they indiciate the company's financial health to the company and to parties interested in the business. 
In the FA exam, business transactions can be classified into i.) Cash and Credit Sales, ii.) Sale Returns, iii.) Cash and Credit Purchases, iv.) Purchase returns and v.) Petty Cash Transactions.

\subsubsection{Cash and Credit Sales}

A sale is the process of trading goods and services for money. A cash sale is a immediate transaction where goods and services are immediately exchanged for money while a credit sale refers to a future transaction where goods and services are exchanged for a future payment. Customers who purchased on credit should repay the business at the end of the current credit term. Customers who receive damaged or incorrect goods can return
them to the business. This is a sales return. When this occurs the business writes a credit note to reduce the value on the sales invoice. 

\subsubsection{Cash and Credit Purchases}

A purchase is the process of exchanging money for goods and services. A cash purchase is an immediate transaction where money is immediately exchanged for goods and services while a credit purchase is a future trasaction where goods and services are immediately received and money is exchanged at a later date. At the end of the credit term and credit purchase must be repaid in full. If the business receives damaged or incorrect goods they
can return them, with the supplier issuing a credit note to reduce the value on the preivously issued sales invoice. 

\subsection{Sales Tax} 

Sales tax is an indirect tax charged on goods and services sold or purchased, they are imposed by the government and tax authorities. Sales tax can be split into two classes, input tax and output tax. 

\begin{itemize}
    \item Input tax is sales tax charged on sales to customers 
    \item Output tax is the sales tax charged on purchases of goods and services.
\end{itemize}

If Output Tax $>$ Input tax, sales tax is considered a current liability and is owed to tax authorities.
If Input Tax $>$ Output tax, sales tax is considered a current asset, with the tax authorities owing the business. 


\subsection{Discounts Received and Allowed}

A discount is a reduction in the prices of goods, or services, below the price at which they would typically cost. Discounts can be classified into trade discounts 
or settlement discounts. Trade discounts are guaranteed discounts, and customers are expected to take advantage of these discounts. Therefore trade discounts offered are always 
considered when recognising transactions. Trade discounts on sales are called discounts received, while trade discounts on purchases are called discounts allowed.

\subsection{Bank Reconciliation}

A bank reconciliation is the process of indentifying and eliminating discrepencies between the bank statement's balance and the balance of the bank ledger account to ensure that both balances agree. 


A standing order is an instruction from the payer to the bank to pay a fixed amount on a predetermined date a third party. 
