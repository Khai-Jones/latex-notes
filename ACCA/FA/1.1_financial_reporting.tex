\section{Financial Reporting and The Financial Statements}

This chapter introduces three foundational topics. Business Entities, Financial Reporting and Users of Financial Statements.

\subsection{Business Entities} 

\subsubsection{Types of Business Entites}

Busniesses exist to make a profit by producing and selling goods and services. Although businesses 
share a similar purpose, they can vary by size, financing and legal standing. These variations affect what is reported in their financal statements. 
A business can operate as a sole trader, partnership or sole trader. The legal differences of each of these business types will fall under 5 different categories. 

\begin{itemize}
    \item Business Continuity 
    \item Changes in Business Ownership 
    \item Taxation 
    \item Legal Action 
    \item Ownership of Business Property
\end{itemize}

\subsubsection{Sole Trader}

A sole trader is an individual that owns, operates and manages a business alone. 

Legally a sole trader faces. 

\begin{itemize}
    \item Business Continuity: A sole trader's business ceases to exist if he/she exits it. It cannot continue unless sold to another individual.  
    \item Changes in Business Ownership: In order for a sole trader's business to change ownership, the existing sole trader must sell their business to the new owner. Since the sole trader is the only person in charge, this decision is made without involving others. 
    \item Taxation: The sole trader is personally taxed on all of the business's profit. The tax authorities do no separate the business from it's owner.  
    \item Legal Action: The sole trader is liable for any penalties or fines incurred if a legal dispute occurs and a court judgement is made against the business.   
    \item Ownership of Business Property: The sole trader owns all of the business's property. 
\end{itemize}

Advantages of a Sole Trader: 

A sole trader has flexibility, control and realtively simple business formalities when conducting their busienss. A sole trader has complete control over their buisness. This entails that the have full ownership over the business's assets and 
is entitled to all profit generated by the business. It is also easy for a sole trader to set up and operate a business. There is no requirment to comply with business regulation and no need to publish financial statements. However, the sole trader may still need 
to publish financial statements to provide information to the tax authorities. Lastly a sole trader can choose how to operate a business and is not bound by any workload requirements. 

Disadvatages of a Sole Trader: 

A sole trader is burden by exposure to liability, difficulty in raising finance and business contination. A sole trader is liable for all of the business's debts. Any personal possesions may have to be sold to pay off the business's debts. 
\subsubsection{Partnership}

\subsubsection{Limited Liability Company}

\subsection{Financial Reporting}

\subsection{Users of Financial Statements}
