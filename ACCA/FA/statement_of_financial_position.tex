\section{Statement of Financial Position (SFP)}

\begin{quote}
\textbf{Definition:} The \textbf{Statement of Financial Position} (SFP) is a primary statement that shows the financial position of a business at a point in time. This includes the assets owned, liabilities owed, and the capital/equity balance. The Statement of Financial Position is sometimes referred to as the ``Balance Sheet''.
\end{quote}

The ``financial position'' can be defined as a company's net worth (Assets less liabilities). The statement of financial position shows the book value or \textbf{carrying amount} at a particular date of the entity, presenting its:
\begin{itemize}
    \item Assets (resources controlled)
    \item Liabilities (obligations owed)
    \item Owners' Capital or Equity (how the business is financed)
\end{itemize}

IAS 1 \textit{Presentation of Financial Statements} states that the statement of financial position is required to have the following items:
\begin{itemize}
    \item Property, plant and equipment
    \item Intangible assets
    \item Inventories
    \item Trade and other receivables
    \item Cash and cash equivalents
    \item Trade and other payables
    \item Provisions
    \item Current tax liabilities
    \item Share capital and reserves
\end{itemize}

Like the statement of profit or loss and other comprehensive income, the statement of financial position of a sole trader would be different from that of a limited company. The sole trader's version of the SFP follows the same principles but has more detail presented and a different capital/equity section.

\subsection*{Example 7: Sole Trader vs. Company Statement of Financial Position}
This example shows the statement of financial position for a sole trader business and a company.

\begin{tabular}{ll}
\toprule
\textbf{Sole Trader Statement of Financial Position} & \textbf{Company Statement of Financial Position} \\
\midrule
\textbf{Non-current assets} & \textbf{Non-current assets} \\
Property, Motor Vehicles (Cost, Depreciation, Carrying value broken down for each) & Property, plant and equipment (total) \\
 & Intangibles (total) \\
\midrule
\textbf{Current assets} & \textbf{Current assets} \\
Inventories (least liquid, always first) & Inventories \\
Receivables (detailed with allowance for receivables, prepayments) & Trade and other receivables (single balance) \\
Cash at bank and in hand (includes petty cash, savings; overdraft separate) & Cash and cash equivalents (single balance; overdraft separate) \\
\midrule
\textbf{Capital} & \textbf{Equity} \\
Capital brought forward & Share capital (always first) \\
Profit for the year & Retained earnings \\
Less: drawings & Other reserves \\
\midrule
\textbf{Non-current liabilities} & \textbf{Non-current liabilities} \\
Bank loan (each liability shown) & Loan capital and other non-current liabilities (detailed in notes) \\
\midrule
\textbf{Current liabilities} & \textbf{Current liabilities} \\
Payables (each balance shown, e.g., Provision, Accruals, Sales tax payable) & Total current liabilities (detailed in notes) \\
\bottomrule
\end{tabular}

\subsection*{Description of SFP Components:}
\begin{itemize}
    \item \textbf{Non-current assets:} Assets purchased for use within the business to generate profits for more than 12 months.
    \begin{itemize}
        \item \textbf{Sole Trader:} Each asset category (e.g., Property, Motor Vehicles) shown separately with cost, accumulated depreciation, and carrying amount.
        \item \textbf{Company:} One total for property, plant and equipment, and one total for intangibles, with detail in notes.
    \end{itemize}
    \item \textbf{Inventories:} Value of goods held for sale or use in manufacturing.
    \begin{itemize}
        \item \textbf{Sole Trader:} May show breakdown in the SFP itself.
        \item \textbf{Company:} Detailed breakdown in a note to the SFP.
    \end{itemize}
    \item \textbf{Liquidity ranking of current assets:} Inventory is always presented first (least liquid), followed by receivables and cash (most liquid).
    \item \textbf{Receivables:}
    \begin{itemize}
        \item \textbf{Sole Trader:} Shows receivables balance, allowance for receivables, and other receivables (e.g., prepayments) separately.
        \item \textbf{Company:} Trade and other receivables included as a single balance. Trade receivables are amounts owed by credit customers and allowance against them; other receivables include prepayments and other income receivables.
    \end{itemize}
    \item \textbf{Cash at bank and in hand / Cash and cash equivalents:}
    \begin{itemize}
        \item \textbf{Sole Trader:} Bank account, petty cash. Savings account might be the extent.
        \item \textbf{Company:} Various balances falling under cash and cash equivalents (details in Statement of Cash Flow chapter). Single balance in SFP, details in notes. Overdraft never included here; shown separately in current liabilities.
    \end{itemize}
    \item \textbf{Capital brought forward / Share capital:} The most significant difference is the capital (sole trader) or equity (company) section.
    \begin{itemize>
        \item \textbf{Sole Trader:} Shows breakdown of capital brought forward + profit for the year - drawings.
        \item \textbf{Company:} Replaced with share capital, retained earnings, and other reserves. Share capital always presented first, followed by reserves.
    \end{itemize}
    \item \textbf{Non-current liabilities:} Liabilities settled (paid) in more than 12 months (e.g., long-term loans).
    \begin{itemize}
        \item \textbf{Sole Trader:} Each liability shown in the SFP.
        \item \textbf{Company:} Loan capital and other non-current liability categories (e.g., provisions), detailed breakdown in notes.
    \end{itemize}
    \item \textbf{Current liabilities:} Liabilities settled (paid) in less than 12 months (e.g., overdraft, trade payables, accruals).
    \begin{itemize>
        \item \textbf{Sole Trader:} Each balance shown on the SFP.
        \item \textbf{Company:} One total shown, content disclosed in notes.
    \end{itemize}
\end{itemize}

\subsection{Interrelationship between SPL\&OCI and the SFP}

There are two sides to every accounting entry, explaining the links between the Statement of Profit or Loss and Other Comprehensive Income (SPL\&OCI) and the Statement of Financial Position (SFP). Several adjustments may affect both statements:
\begin{itemize}
    \item \textbf{Profit for the year:} Transferred to the equity section of the Statement of Financial Position.
    \item \textbf{Gain on revaluation of Property, plant and equipment:} Recorded in the Statement of Financial Position (DR PPE), with a corresponding credit to Revaluation surplus (SPL\&OCI).
    \item \textbf{Depreciation charged:} Recorded in both statements (DR Depreciation Expense in SPL\&OCI, CR PPE – Accumulated Depreciation in SFP).
    \item \textbf{Irrecoverable debt:} When written off (SPL), it reduces the receivables balance (SFP).
    \item \textbf{Sale and assets:} When a sale (SPL) is recorded, it creates a receivables balance or increases cash (SFP).
    \item \textbf{Purchase and liabilities:} When a purchase (SPL) is recorded, it creates a payables balance or decreases cash (SFP).
    \item \textbf{Closing inventory:} When closing inventory (SPL) is recorded, it is also recorded in Current assets (SFP).
    \item \textbf{Allowance for irrecoverable debts:} When the allowance for irrecoverable debts (SFP) balance is adjusted, the opposite entry goes to irrecoverable debts expense (SPL).
\end{itemize}

\subsection*{Activity 2: Matching SPL\&OCI and SOFP Entries}
Match the SPL\&OCI entries on the left-hand side with their corresponding SFP entry on the right.
\begin{itemize}
    \item Income tax expense $\rightarrow$ Current tax liability
    \item Other comprehensive income $\rightarrow$ Revaluation surplus
    \item Closing inventory in cost of sales $\rightarrow$ Closing inventory in current assets
    \item Irrecoverable debt written off $\rightarrow$ Receivables
    \item Irrecoverable debt-movement in allowance for receivables $\rightarrow$ Allowance for receivables
    \item Depreciation expense $\rightarrow$ Accumulated depreciation
    \item Movement in provision $\rightarrow$ Provisions
\end{itemize}

\subsection{Reserves}

The main difference in the format of an SFP for a sole trader and company is in the capital or equity section.
The sole trader's Statement of Financial Position shows the breakdown of the capital balance as capital brought forward + profit for the year – drawings. The single owner of the business owns the resulting closing balance.
However, a company may have many owners or ordinary shareholders rather than a single owner. The amounts that they own are summarised in share capital and reserves.

Reserves are balances representing gains or losses belonging to the business owners. They can be split into two types:
\begin{itemize}
    \item \textbf{Statutory reserves:} Reserves that a company must set up by law and are not available to be distributed as dividends (e.g., Share premium).
    \item \textbf{Non-Statutory reserves:} Made up of profits that can be distributed as dividends (e.g., Retained earnings).
\end{itemize}
These reserve balances are shown along with share capital in the equity section of the statement of financial position.

\begin{itemize}
    \item \textbf{Retained Earnings:} The main reserve of a business entity, representing accumulated post-tax profits. It is a distributable reserve which can be used to pay a dividend to ordinary shareholders.
    \item \textbf{Share Premium:} Created when new shares are issued at a price above their par value. Recognized as part of capital, it is a non-distributable reserve (cannot pay dividends) but can be used to fund a bonus issue of shares.
    \item \textbf{Revaluation Surplus:} Represents the increase in value of property, plant and equipment. It is an unrealised gain (asset not sold), recognized as part of capital and is a non-distributable reserve (cannot pay dividends).
    \item \textbf{Other Reserves:} Can be created for various uses and in line with accounting standards, but are outside the scope of this course.
\end{itemize}

\subsection*{Example 8: Statement of Changes in Equity (Tishla Co)}
Tishla Co had the following balances at 1 January 20X8:
\begin{itemize}
    \item Share capital \$10,000
    \item Share premium \$3,000
    \item Retained earnings \$54,860
    \item Revaluation surplus \$6,000
\end{itemize}
Tishla Co made a profit of \$12,800 for the year, recognised a revaluation surplus of \$4,000 and paid a dividend of \$2,000. In addition, a 1-for-5 bonus issue of shares was made from the share premium account.

The completed statement of changes in equity is as follows:

\begin{tabular}{lrrrrr}
\toprule
 & \textbf{Share capital (\$)} & \textbf{Share premium (\$)} & \textbf{Revaluation surplus (\$)} & \textbf{Retained earnings (\$)} & \textbf{Total (\$)} \\
\midrule
At 1 January 20X8 & 10,000 & 3,000 & 6,000 & 54,860 & 73,860 \\
Dividends & & & & (2,000) & (2,000) \\
Total comprehensive income for the year & & & 4,000 & 12,800 & 16,800 \\
Issue of share capital (Bonus Issue) & 2,000 & (2,000) & & & \\
\midrule
\textbf{At 31 December 20X8} & \textbf{12,000} & \textbf{1,000} & \textbf{10,000} & \textbf{65,660} & \textbf{88,660} \\
\bottomrule
\end{tabular}

\section{SFP – Disclosures}

\subsection*{Purpose of Disclosures}
Disclosures in the Statement of Financial Position are essential for users of financial statements to understand the financial information. The SFP for a company typically includes only totals, with all detail provided in disclosure notes.
The purpose of financial statements is to provide information about a business entity's financial position, financial performance, and cash flows. A full set of disclosures is required to meet the qualitative characteristic of understandability.

\subsection*{Required Disclosures for SFP Items:}
Disclosures are required for the following items under the statement of financial position (specific requirements are detailed in their respective chapters):
\begin{itemize}
    \item Property, plant and equipment
    \item Intangible assets
    \item Provisions
    \item Events after the reporting period
    \item Inventories
\end{itemize}

Similar to preparing the Statement of Profit or Loss, the individual assets, liabilities, and capital ledger accounts from the final trial balance are transferred and presented in the Statement of Financial Position. The statement of financial position is the final product of the accounting system.

\subsection*{Example 9: Sole Trader Statement of Financial Position (Cake Catering)}
Below shows the final trial balance of Cake Catering and the completed Statement of Financial Position. Each asset, liability, and capital ledger account balance is transferred to the statement.

\textbf{Cake Catering Trial Balance for the year ended 31 December 20X2}

\begin{tabular}{lrr}
\toprule
\textbf{Account} & \textbf{DR (\$)} & \textbf{CR (\$)} \\
\midrule
Property at cost & 145,250 & \\
Property – accum depn at 31 Dec 20X2 & & 31,955 \\
Motor vehicles at cost & 25,420 & \\
Motor vehicles – accum depn at 31 Dec 20X2 & & 9,785 \\
Computer and office equipment – at cost & 12,510 & \\
Computer and office equipment – acc. depn at 31 Dec 20X2 & & 5,798 \\
Shop fixtures and fittings – at cost & 33,841 & \\
Shop fixtures and fittings - acc. depn at 31 Dec 20X2 & & 11,463 \\
Inventory – opening at 1 Jan 20X2 & 37,412 & \\
Receivables control account & 35,091 & \\
Allowance for irrecoverable debts at 31 Dec 20X2 & & 4,750 \\
Prepayments at 31 Dec 20X2 & 8,450 & \\
Cash at bank & 10,674 & \\
Capital account at 1 Jan 20X2 & & 172,127 \\
Drawings & 25,410 & \\
Capital introduced & & 4,000 \\
Bank loan & & 26,950 \\
Provision at 31 Dec 20X2 & & 800 \\
Payables control account & & 36,741 \\
Accruals at 31 Dec 20X2 & & 6,610 \\
Sales tax payable & & 1,473 \\
Sales & & 608,989 \\
Purchases & 420,974 & \\
Rent & 26,700 & \\
Wages & 86,724 & \\
Finance cost & 1,693 & \\
Telephone, postage and stationery & 2,777 & \\
Other operating expenses & 29,130 & \\
Depreciation expense & 15,176 & \\
Irrecoverable debt expense & 4,209 & \\
\textbf{Total} & \textbf{921,441} & \textbf{921,441} \\
\bottomrule
\end{tabular}

\textit{Note: The closing inventory of Cake Catering is \$39,125.}


\newpage 

\textbf{Cake Catering Statement of Financial Position as at 31 December 20X2}

\begin{center}
\begin{tabular}{lrrr}
\toprule
 & \textbf{Cost (\$)} & \textbf{Depreciation (\$)} & \textbf{NBV (\$)} \\
\midrule
\textbf{Non-current assets} & & & \\
Property & 145,250 & 31,955 & 113,295 \\
Motor Vehicles & 25,420 & 9,785 & 15,635 \\
Computer and office equipment & 12,510 & 5,798 & 6,712 \\
Shop fixtures and fittings & 33,841 & 11,463 & 22,378 \\
\addlinespace
Total NCA & 217,021 & 59,001 & 158,020 \\
\midrule
\textbf{Current assets} & & & \\
Inventory & & & 39,125 \\
Receivables & 35,091 & & \\
Less: allowance for irrecoverable debts & & (4,750) & 30,341 \\
Prepayments & & & 8,450 \\
Cash at bank and in hand & & & 10,674 \\
\addlinespace
Total Current Assets & & & 88,590 \\
\midrule
\textbf{Total assets} & & & \textbf{246,610} \\
\midrule
\textbf{Capital} & & & \\
Capital brought forward & & & 172,127 \\
Capital introduced & & & 4,000 \\
Profit for the year & & & 23,319 \\
Less: drawings & & & (25,410) \\
\addlinespace
Total Capital & & & 174,036 \\
\midrule
\textbf{Non-current liabilities} & & & \\
Bank loan & & & 26,950 \\
\midrule
\textbf{Current liabilities} & & & \\
Provision & & & 800 \\
Payables & & & 36,741 \\
Accruals & & & 6,610 \\
Sales tax payable & & & 1,473 \\
\addlinespace
Total Current Liabilities & & & 45,624 \\
\midrule
\textbf{Total Capital and Liabilities} & & & \textbf{246,610} \\
\bottomrule
\end{tabular}
\end{center}

