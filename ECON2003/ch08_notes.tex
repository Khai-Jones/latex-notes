\noindent Economic Growth is measured by using data on gross domestic product ( GDP ) which also measures total income of everyone in a country. 
The objective of this unit is to explain the differences in total income across countries. Output( GDP and National Income ) is determined by a country's labor, capital
and technology. Differences in output can then be explained by the differenecs between nation's labor, capital, and technology. 
\\[5pt]
The Solow Model is designed to be a methodology that dynamically explains why a nation's output ( GDP and National Income ) grows and why some economies grow faster, slower or not at all. To do this 
the Solow Model introduces, savings ( invesments ) , technological advancements and population growth as key determinants/factors that alter a nation's output "over time".

\section*{Components: The production and consumption function} 

\noindent The Supply of Goods in The Solow Model comes from the production function: 
\\[5pt]
Pre-Assumptions:

\itenum{ 
    \item The production function is assumed to exibit constant returns to scale.
    \item The production function is assumed te exibit marginal diminishing returns. 
    } 


\al{ 
   zY = F(zK, zL)
}

\noindent This assumption allows us to explain a nation's GDP relative to the number of workers in the nation's economy. For the Solow Model $z$ is said to be $\frac{1}{L}$, so that the production function becomes: 

\al{
    Y/L = F(K/L, L/L) \\[5pt]
    Y/L = F(K/L, 1) 
}

\noindent To simplify the equation to avoid congnative overload, the per-worker($1/L$) denoations become:  
\itenum{
\item $Y/L$ = $y$
\item $F$ = $f$
\item $(K/L, 1)$ = $k$
}

\al{ 
    y = f(k)
}

\noindent The Demand of Goods in the Solow Model come from the consumption function. 
\\[5pt]
Pre-Assumptions: 

\itenum{
\item Workers save a fraction of their income, based on a saving rate (s). 
}

\noindent In the Solow Model, $y$ ( GDP and Nation's Income ) is divided between $c$ consumption per worker and $i$ investment per worker: 

\al{
    y = c + i 
}

\noindent This is the per worker version of the national income accounts identity. Since persons save a fraction of their income before consumption, $c$ can be represented as $(1-s)y$. Inserting this into the above equation we get. 

\al{
    y = (1-s)y + i
}

\noindent Rearranging for Investment: 

\al{
    y = (1-s)y + i \\ 
    y - (1 -s)y = i \\
    y(1 - 1 + s) = i \\ 
    ys = i 
}

\noindent Thus the saving rate not only determines consumption but the rate which investment are made. With these assumption we have enough information 
to go deeper into the Solow Model. The production function $y = f(k)$ determines the nation's output and the saving rate $s$ determines how ouput is
allocated between consumption and investment. 

\section*{Growth in the Capital Stock and The Steady State} 


Given that we have estabilished the production function has $y = f(k)$. The nation's capital output determines the nation's output, so changes in the nation's capital output cause's changes in the nation's output. Thus we 
must look at what factors affect capital stock. This is where investment and depreciation come into the fray. Investment causes capital to rise and Depreciation causes captial to decrease. 

\noindent Investment: 

\al{
    i = sy \\ 
    i = sf(k) 
}

\noindent When $y$ is swicthed out for $f(k)$ the equation implies that the current amount of capital in the economy directly increases the amount of new capital. Now to fit depreciation in the model, we 
assume that a fixed amount of capital wears out each year and is denoted by $\delta{k}$ where $\delta =$ the depreaction rate and $k =$ capital. The change in capital can be represented mathematically as: 

\al{
    \Delta{k} = i - \delta{k} \\
    \Delta{k} = sf(k) - \delta{k} 
}

\noindent Where:
\itenum{
\item $\Delta{k} =$ The change in capital between one year and another.  
}

\subsection*{Steady State of capital $k^*$} 

\noindent The steady state level of capital is the level at which investment exactly balances depreciation. At this level, there is no change in capital as there is no change in investment or depreciation. 
:This level is siginificant because $1.$ an economy that is already at this level will not move from it and $2.$ If an economy is not at this level it will eventually move towards it, thus becoming/representing the long-run equilibrium of the economy.

\subsection*{How Saving Affects Growth} 

When the economy increases it's savings rate, it leads to an increase in the steady state. Why? When $s_1$ moves to $s_2$ it leads to an incease in investment [$sf(k)$]. Since investment is now greater than depreciation, the steady state begins to rise until 
depreciation equally matches investment. Economic growth temporarily increases, but it is not sustained, meaning that a change in savings causes an 'level effect' and not a 'growth effect'. 
A high savings rate means a high steady state and and low savings rate means a low steady state.


\section{The Golden Rule Level of Capital}

\subsection{Comparing Steady States}


