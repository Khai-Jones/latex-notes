\newpage 

\section*{Chapter 8 Questions}
\itenum{
    \item In the Solow model, how does the saving rate affect thesteady-state level of income? How does it affect the steady-state rate of growth?
\item Why might an economic policymaker choose the Golden Rule level of capital?
\item Might a policymaker choose a steady state with more capital than in the Golden Rule steady state? Might a policymaker choose a steady state with less capital than in the Golden Rule steady state? Explain your answers.
}


\subsection*{Question 1}

\al{ 
    Y = F(K, l) = K^{\frac{1}{3}}L^{\frac{2}{3}}
}
 Country A and Country B both have the production function. Does this production function have constant returns to scale? Explain. 
      What is the per-worker production function y = f(k)? Assume that neither country experiences population growth or technological progress an that 20 percent of capital depreciates each year. 
      Assume further that country A saves 10 percent of output each year and country B saves 30 percent of output each year. Using your answer from part (b) and the steady-state condition that investment equals 
      depreciation, find the steady-state level of capital per worker for each country. Then find the steady-state levels of income per worker and consumption per worker. Suppose that both countries start off with a capital stock 
      per worker of 1. What are the levels of income per worker and consumption per worker? Remmebering that the change in the capital stock is investment less depreciation, use a calculator to show how the capital stock per 
      worker will evolve over time in both countries. For ecah year, calculate income per worker and consumption per worker. How many years will it be before the consumption in country B exceeds the consumption in country A? 


      \subsection*{Question 4}

      "Devoting a larger share of national output to investment would help restore rapid productivity growth and rising living standards." 
      Do you agree with this claim? Explain, using the Solow Model. 

      \al{
          i = s \cdot f(k) 
      }

     \noindent The Solow Model explains long-run economic growth through capital accumulation, technological progress and population growth. I partially agree with the previously stated claim. If a larger share of national output 
     is allocated to investment, it ofsets the balance with depreciation and leads to an increase in the economy's steady state causing an 'level effect'. This growth is temporary as only changes in technological progress leads to 
     a change in the growth rate.
